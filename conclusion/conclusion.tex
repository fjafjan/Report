The stated goal of the thesis was to verify the theoretical predictions of Yarin, Hinch and Leal \cite{Yarin, Leal} and 
to verify that the condition of creeping flow is met by examining the time reversibility of the dynamics. 

The match of the data to theory is in some cases very good and the winding number matching for different initial 
conditions confirm this as well. Further more these different initial conditions exhibit very different behaviours, 
either being almost constant to being highly quasi-periodic. We can then conclude to have managed to measure Jeffery 
orbits as predicted by Yarin, Hinch and Leal \cite{Yarin, Leal}. 

The time reversibility is unreliable, for many measurements it reverses very well, and for many other it reverses quite 
poorly. We have not understood well what causes this type of unpredictability that can occur for the same particle only 
minutes apart. It is possible that time reversal can be improved by attempting to somehow increasing the viscosity 
without changing the optical index of the liquid or by slowing down reversals further. However considering what appears 
to be significant expansion and contraction of the channel perhaps using a more solid material might produce better 
results. 

The automated tracking was somewhat useful, but with the increased speed of measurements as well as smaller particles 
there needs to be a better predictive model to not lose the particle in reversals. It is also less needed as 
measurements take significantly less time, but if significantly slower reversals are the preferred way of improving 
time reversibility then it could be more relevant. 