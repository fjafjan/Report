the Jeffery Orbits are frequently used across scientific fields but thus far there has been few experimental studies 
of the orientational dynamics and in particular the quasi-periodic and chaotic orbits have only been studied experimentally
by Einarsson \emph{et al}\cite{JonasExperiment}\cite{Mishra}.
The goal of the thesis was to verify the theoretical predictions of Yarin, Hinch and Leal \cite{Yarin, Leal} and to show that the same particle could exhibit quasi periodic and periodic behaviour for different initial conditions.

There are several measurements that agree very well with theoretical models using both the phase map matching the winding number estimation. Further more different initial conditions exhibit very different behaviours across different particles, 
either being almost constant to being quasi-periodic with a large regular variation. We can then conclude to have managed to measure the
orbits predicted by Yarin, Hinch and Leal \cite{Yarin, Leal}.

The time reversibility is unreliable, for some measurements the dynamics revert very well, and for others does not. 
We have not understood well what causes this type of unpredictability that can occur for the same particle only 
minutes apart. The primary focus of future efforts should go into understanding and correcting whatever issues causes the 
reversals to be so unreliable. It is possible that time reversal can be improved by attempting to somehow increasing the viscosity 
without changing the optical index of the liquid or by slowing down reversals. Another possibility is limiting the 
significant expansion and contraction of the channel that seems to occur.

The automated tracking was useful, but with the increased speed of measurements as well as smaller particles 
there needs to be a better predictive model to not lose the particle in reversals. It is also less needed as 
measurements take significantly less time, but if significantly slower reversals are the preferred way of improving 
time reversibility then it could be more relevant. 