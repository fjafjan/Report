The Jeffery orbits are frequently used across scientific fields but thus far there has been few experimental studies 
of the orientational dynamics. In particular the quasi-periodic and chaotic orbits have only been studied experimentally
by Einarsson \emph{et al.}~\cite{JonasExperiment} and Mishra \emph{et al.}~\cite{Mishra}.
The goal of the thesis was to verify the theoretical predictions of Yarin \emph{et al.}~\cite{Yarin} and Hinch,Leal~\cite{Leal} and to show that the same particle could exhibit quasi-periodic and periodic behaviour for different initial conditions.

% % Introduce the experiment

There are several measurements that agree well with theoretical models by comparing using both the phase map matching and the winding number estimation. Furthermore, for two particles we have found that different initial conditions exhibit different behaviour, 
from almost constant to being quasi-periodic with a large regular variation. We can therefore conclude that we have measured the
orbits similar to those predicted by Yarin \emph{et al.}~\cite{Yarin} and Hinch, Leal~\cite{Leal}.

The time reversibility of the system is unreliable. For some measurements the dynamics revert very well, and for others they does not. 
We have not understood fully what causes this type of unpredictability that can occur for the same particle only 
minutes apart. The primary focus of future efforts should be understanding and correcting whatever issues cause the 
reversals to be so unreliable. It is possible that the time reversal can be improved by increasing the viscosity of the liquid
without changing the optical index. Another or by slowing down reversals. Another possibility is limiting the 
significant expansion and contraction of the channel that seems to occur.

The automated tracking was useful in earlier measurements, but with the use of smaller particles, as well as increased flow speed during measurements, 
a better predictive model is needed to not lose the particle that is tracked during reversals. Automated tracking is also less urgent as
measurements take less time. If significantly slower reversals are used to improve time reversibility then the automated tracking will be a higher priority.