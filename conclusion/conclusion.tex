The Jeffery orbits are frequently used across scientific fields \cite{Tolga}\cite{geology} but thus far there has been few experimental studies 
of the orientational dynamics. In particular the quasi-periodic and chaotic orbits have only been studied experimentally
by Einarsson \emph{et al.}~\cite{JonasExperiment} and Mishra \emph{et al.}~\cite{Mishra}.
The goal of the thesis was to verify the theoretical predictions of Yarin \emph{et al.}~\cite{Yarin} and Hinch,Leal~\cite{Leal} and to show that the same particle could exhibit quasi-periodic and periodic behaviour for different initial conditions.

% % Introduce the experiment

There are several measurements that agree well with theoretical models by comparing using both the phase map matching and the winding number estimation. Furthermore, for two particles we have found that different initial conditions exhibit different behaviour, 
from almost constant to being quasi-periodic with a large regular variation. However we make the assumption that the asymmetry of the broken rods can be compared to that of the triaxial particles without large errors. We can therefore conclude that if this assumption is true, that we have measured the
orbits predicted by Yarin \emph{et al.}~\cite{Yarin} and Hinch, Leal~\cite{Leal}.

We have not found any particles that exhibit chaotic motion. If we again assume that the difference in asymmetry between the broken rods and triaxial particles does not have a major impact, we would not expect to find chaotic orbits for the particles we use. Chaotic orbits are only common once $\epsilon \geq 0.2$~\cite{AntonThesis} and the orbits measured match $\epsilon \leq 0.05$. 

The time reversibility of the experiment is unreliable. For some measurements the dynamics of the particle revert very well, and for others they does not. 
We have not understood fully what causes this type of unpredictability that can occur for the same particle only 
minutes apart. The primary focus of future efforts should be understanding and correcting whatever cause the 
reversals to be so unreliable. It is possible that the time reversal can be improved by increasing the viscosity of the liquid, or by slowing down reversals. Another possibility is limiting the expansion and contraction of the channel that seems to occur.

The automated tracking was useful in earlier measurements, but with the use of smaller particles, as well as increased flow speed during measurements, 
a better predictive model is needed to not lose the particle that is tracked during reversals. Automated tracking is also less urgent as
measurements take less time. If significantly slower reversals are used to improve time reversibility then the automated tracking will be a higher priority.