\section{Introduction}
My goal in this thesis is to study and better understand the dynamics of ellipsoidal particles in shear flows. The thesis is a continuation of two previous MSc theses~\cite{AntonThesis, JonasThesis}. The methodology is to experimentally measure the orientational dynamics of $\mu$m-sized glass particles in a shear flow, then comparing the results to results from theoretical models. In the first part of the thesis, I describe the improvements that were made to the experimental setup, in particular an automated tracking system. In the second part of the thesis, the measurements and their analysis is discussed. But before discussing either of these subjects more in depth some background is needed.

\section{Background}
Understanding the orientational dynamics of particles in flow might appear somewhat esoteric to someone unfamiliar with the topic, but there are a number of fields where it is very useful. In medical applications understanding the dynamics of ellipsoidal particles such as bacteria can be relevant to a detailed understanding of their interactions with cells and other bodies~\cite{Tolga}. In geology the rotational dynamics of mineral grains in magma flows determines rock structures which form after the magma cools~\cite{geology}.

One of the first papers to show the importance of understanding particle dynamics in flow was by Einstein in 1905~\cite{Einstein}. He showed how the rotation of spherical particles suspended in a liquid would increase the viscosity. Jeffery in his 1922 paper~\cite{Jeffery} extended these results to ellipsoidal  particles and for systems where intertial effects could be disgarded he derived equations for the orientational dynamics of the particles. He also solved the equations for axisymmetric particles and found their motion to be periodic and depending only on the initial condition of the particle. 

Investigation of triaxial particles was started by Gierszewski \& Chaffey~\cite{Chaffey} and was continued by Hinch \& Leal~\cite{Leal} and more recently by Yarin \emph{et al.}~\cite{Yarin}. 
The dynamics Jeffery had found for axisymmetric particles was periodic, but it was shown by Hinch \& Leal that for triaxial particles some orbits could be doubly periodic, in other words exhibiting two separate independent periods. This behaviour is referred to as \emph{quasi-periodic} in this thesis.

Yarin \emph{et al.} used numerical simulations to generate a surface of section~\cite{SurfaceOfSection} for ellipsoidal particles with different shapes. They showed that not only were there doubly periodic or quasi-periodic orbits but when the ellipsoids had sufficiently different minor and median axes there could be chaotic orbits. In these orbits the particles rotate periodically perpendicular to the flow but the orientation of the particles in the flow changes unpredictably over longer time scales. % Change or remove

Several other surfaces of section were produced by Johansson ~\cite{AntonThesis} using the same method as Yarin \emph{et al.} It was shown that even small asymmetries of the order of 1\% lead to quasi-periodic motion for some initial conditions.

Attempts to experimentally verify the theoretical results of Jeffery were initially performed by Goldsmith and Mason in 1962~\cite{Mason} who used flow in a glass pipe to observe the rotation rate of particles of several different shapes. They confirmed that the rotation rate matched well with that predicted from Jeffery orbits but they did not study the actual orbits. Since then most experimental research, such that as by Harlen and Koch~\cite{fibersspension}, has focused on how dilute suspensions of particles affect the properties of a liquid. Only tangential efforts such as by Tolga~\cite{Tolga} were concerned with the Jeffery orbits. An excellent summary of both theoretical and experimental results was written by Petrie~\cite{Petrie} in 1999.

Dedicated experiments to measure the actual Jeffery orbits in angular components and verify the orientational dynamics were performed by Einarsson \emph{et al.} ~\cite{JonasExperiment}. They studied the rotational dynamics of $\sim 30 \mu$m long polymer particles in a microfluidic channel. Some particles showed orbits that were shown to be qualitatively similar to some Jeffery orbits. These particles also very closely retraced their motion when reversing the flow implying that no significant amount of noise had disturbed their orientational motion nor had there been any inertial effects. the vast majority of particles were asymmetric to the degree that their orbits were chaotic or highly quasi-periodic. No particular particle could also be shown to exhibit both quasi-periodic and periodic motion. Moreover the width and length of particles varied greatly and could not be measured accurately.

% I really want a cite for thus but how could I possibly do that.
The goal of this thesis is to improve the setup from Johansson~\cite{AntonThesis} and Einarsson \emph{et al.}~\cite{JonasExperiment} in order experimentally verify the results of Yarin \emph{et al}.~\cite{Yarin} and Hinch, Leal~\cite{Leal}. The aim is to show that the same particle may exhibit different types of motion for different initial conditions: Periodic motion, quasi-periodic motion and chaotic motion. Furthermore that particles with different asymmetries may show different motion for the same initial conditions. To this end we observe the orientation of a $\mu$meter length particle in a creeping shear flow. $\mu$m sized particles guarantee that there are no inertial effects and are also small enough to be controlled by optical tweezers. The flow is shown to be creeping by demonstrating that the particle dynamics revert as the flow is reversed. The results are compared to theoretical predictions for different initial conditions and asymmetries.

%doing so would be very important to actually motivating using these results as well as possibly finding the limitations of this theory in real world applications. Understanding the dynamics of ellipsoidal particles in shear flow could be useful for example in our understanding of microscopic bacteria in blood stream 