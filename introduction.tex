


The goal of this thesis is to study and better understand the dynamics of particles in flows, and is a continuation of two previous masters theses \cite{AntonThesis}\cite{JonasThesis}. This is done by analysing micrometer length glass particles in a shear flow and comparing this to theoretical models. Before discussing either of these processes more in depth some background and basic theory is needed.

\subsection{Background}
The study of particle dynamics in flow began with Einsteins paper from 1905 \cite{Einstein} in which he showed that the viscocity of suspended particles would increase the viscocity of the fluid and how much. Jefferey in his 1922 paper \cite{Jeffrey} extended these results to ellipsoidal axis-symmetrical particles but then also studied the orientational dynamics of the particles. Disregarding intertial effects the motion was found to be periodic and depending only on the initial condition of the particle.

Development of tri-axial particle, started with Gierszewski \& Chaffey (1978)\cite{Chaffey} and was continued by Hint \& Leal (1979) and more recently by Yarin et al in 1993\cite{Yarin}. 
Now the dynamics found for axis-symmetric particles by Jeffrey were periodic, but it was shown first by Hinch \& Leal that some orbits would be doubly periodic, in other words following two separate independent periods. 
Yarin then was able to use numerical simulations to generate a surface of section \cite{SurfaceOfSection} for some relations between the two minor axes showing that not only was there double periodic or quasi periodic orbits but for sufficiently large differences in axes there would be chaotic orbits. 
Several other surfaces of were also produced by Johansson (2012)\cite{AntonThesis} with higher resolution thanks to improvements in computing power. It was there shown that only very small asymetries on the order of 1\% will lead to quasi-periodic motion for some initial conditions.

Experimental studies of these theoretical results were first done by Goldsmith and Mason in 1962\cite{Mason}
The most recent experimental work done to verify this was done by Einarsson et al \cite{JonasExperiment} 
The experimental history of verifying these results is more sparse, 


The most extensive work on tri-axial particles was done by Hinch \& Leal in their 1979 paper \cite{Leal}
	
The dynamics of particles in flow has been studied for many years, with early contributions from Einstein in his 1905 paper "Something something Einstein paper"\cite{EinsteinsPaper}, continued by Whatever his name is Jeffrey in 1922 and many since. Jeffrey derived analytically explicit solutions for the motions of triaxial particles in a shear flow called 'Jeffrey Orbits' (see section \ref{theory}) which have been used in various theoretical developments. However there is rather limited experimental evidence that such orbits actually exist \cite{IReallyWantACiteHere}. Previous work by Einarsson\cite{EinarssonsPaper} et al did not satisfactorally show this, presumably because of assymetric particles, possible thermal noise and inaccurate tracking. This thesis has used particles with better symmetry and bla bla bla


A very simple description of the experimental setup is a microfluidic channel where the rotational dynamics of microrods are captured using a camera connected to a confocal microscope. A more detailed description can be found in section \ref{sec:experimental_setup}. 

The focus of this thesis is on improving the experimental setup both in terms of the quality of results as well as the efficiency and ease of gathering such results as well as evaluating the results of these improvements.

The main improvements have been to replace the particles as well as automating the tracking of particles and the flow direction in the channel.