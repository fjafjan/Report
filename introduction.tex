\section{Introduction}
My goal in this thesis is to study and better understand the dynamics of ellipsoidal particles in shear flow. It is a continuation of two previous masters theses \cite{AntonThesis, JonasThesis}. This is done by analysing micrometer length glass particles in a shear flow and comparing this to theoretical models. In the first part of the thesis I will write about the improvements that were made to the experimental setup including an automated tracking. In the second part the meassurements and the analysis of them are discussed. Before discussing either of these processes more in depth some background and theory is needed.

\subsection{Background}
Understanding the dynamics of particles in flow might appear somewhat esoteric but there is actually a number of topics 
where it is important. In fluid dynamics non-newtonian fluids are often high concentrations of particles suspended in 
liquid, such ketchup or blood, and understanding their behaviour might need an improved understanding of how each 
individual particle will behave. In medical applications understanding the dynamics of ellipsoidal particles like 
bacteria can be relevant to a detailed understanding of their interactions with cells and other bodies as discussed by 
Tolga \emph{et al}\cite{Tolga}. 

The study of particle dynamics in flow began with Einstein's in 1905 \cite{Einstein} in which he showed how much the suspended spherical particle would increase the viscosity of a fluid. Jeffery in his 1922 paper \cite{Jeffery} extended these results to ellipsoidal axis-symmetrical ellipsoidal  particles and derived equations for the orientational dynamics of the particles, in other words how the particles would rotate as a function of time. For systems where inertial effects could be discarded the motion was found to be periodic and depending only on the initial condition of the particle. 

Investigation of triaxial particles was started by Gierszewski \& Chaffey\cite{Chaffey} and was continued by Hinch \& Leal\cite{Leal} and more recently by Yarin \emph{et al}\cite{Yarin}. 
The dynamics found for axis-symmetric particles by Jeffrey were periodic, but it was shown by Hinch \& Leal that some orbits would be doubly periodic, in other words following two separate independent periods. This behaviour will here be referred to as quasi-periodic.

Yarin was able to use numerical simulations to generate a surface of section \cite{SurfaceOfSection} for some asymmetric relations between the two minor axes showing that not only was there double periodic or quasi periodic orbits but for sufficiently large asymmetries there would be chaotic orbits. 
Several other surfaces of section were produced by Johansson \cite{AntonThesis} using the same method as Yarin with higher resolution thanks to improvements in computing power. It was shown that even small asymmetries on the order of 1\% will lead to quasi-periodic motion for some initial conditions.

Attempts to experimentally verify these theoretical results were first performed by Goldsmith and Mason in 1962\cite{Mason} who used flow in a glass pipe to observe the rotation rate for several different particle shapes. He was able to confirm that the rotation rate matched well with that predicted from Jeffery orbits but he did not study the actual orbits. Since then most experimental research, such as by Harlen and Koch\cite{fibersspension} has been focused on how diluted suspensions of particles effect the properties of a liquid. Only tangential efforts such as by Tolga\cite{Tolga} were concerned with the Jeffery orbits. A good summary of both theoretical and experimental results was written by Petrie\cite{Petrie} in 1999.

The first dedicated experiments to measure the actual Jeffery orbits in angular components and verify the orientational 
dynamics was done by Einarsson \emph{et al} \cite{JonasExperiment} in 2011. Although there was some promising results, the vast majority of particles were asymmetric to the point of chaotic or highly quasi-periodic. Moreover the width and 
length of particles both varied greatly and could not be measured accurately.
This meant that although the orbits could be qualitatively shown to be similar to some 
Jeffery orbits no particular particle could be shown to exhibit both quasi periodic and periodic motion, and there was 
no particle that could be well matched to a particular orbit. There was also few particles that showed a 
perfectly identical reversal through the entire length of the channel when the flow was reversed.


% I really want a cite for thus but how could I possibly do that.
The goal of this thesis is to experimentally verify the results of Yarin and Hinch, Leal\cite{Yarin, Leal} and show that the same particle will show different types of motion for different initial conditions and furthermore that different particles will also show different motion for the same initial conditions based on the asymmetry. This is done by observing the orientation of a micrometer length particle in a creeping shear flow. The flow is shown to be a creeping by demonstrating that the particle dynamics revert as the flow is reverted. The results for different initial conditions will then be compared to theoretical predictions.

%doing so would be very important to actually motivating using these results as well as possibly finding the limitations of this theory in real world applications. Understanding the dynamics of ellipsoidal particles in shear flow could be useful for example in our understanding of microscopic bacteria in blood stream 