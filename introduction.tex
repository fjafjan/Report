\section{Introduction}
The goal of this thesis is to study and better understand the dynamics of particles in flows, and is a continuation of two previous masters theses \cite{AntonThesis}\cite{JonasThesis}. This is done by analysing micrometer length glass particles in a shear flow and comparing this to theoretical models. Before discussing either of these processes more in depth some background and basic theory is needed.

\subsection{Background}
The study of particle dynamics in flow began with Einstein's paper from 1905 \cite{Einstein} in which he showed that the viscocity of suspended particles would increase the viscosity of the fluid and how much. Jefferey in his 1922 paper \cite{Jeffrey} extended these results to ellipsoidal axis-symmetrical particles but then also studied the orientational dynamics of the particles. Disregarding intertial effects the motion was found to be periodic and depending only on the initial condition of the particle.

Development of triaxial particles, started with Gierszewski \& Chaffey (1978)\cite{Chaffey} and was continued by Hint \& Leal (1979) and more recently by Yarin et al in 1993\cite{Yarin}. 
Now the dynamics found for axis-symmetric particles by Jeffrey were periodic, but it was shown first by Hinch \& Leal that some orbits would be doubly periodic, in other words following two separate independent periods. 
Yarin then was able to use numerical simulations to generate a surface of section \cite{SurfaceOfSection} for some relations between the two minor axes showing that not only was there double periodic or quasi periodic orbits but for sufficiently large differences in axes there would be chaotic orbits. 
Several other surfaces of were also produced by Johansson (2012)\cite{AntonThesis} using the same method as Yarin with higher resolution thanks to improvements in computing power. It was there shown that only very small asymmetries on the order of 1\% will lead to quasi-periodic motion for some initial conditions.

Experimental studies of these theoretical results were first done by Goldsmith and Mason in 1962\cite{Mason} who confirmed that the rotation rate matched well with that predicted from Jeffrey orbits but he did not study the actual orbits. Since then most experimental research, such as CITE SOMETHING HERE has been focused on diluted suspensions of particles and the increased viscocity and other properties this causes. As good summary of both theoretical but primarily experimental results was written by Petrie in 1999 \cite{Petrie}.

The first experimental work to try to measure the actual Jeffrey orbits in their Euler angles was done by Einarsson et al \cite{JonasExperiment} in 2011. Although there was some promising results, the vast majority of particles were asymmetric. Moreover the width and length of particles both varies, meaning that the aspect ratio could not accurately be determined. 

