\section{Introduction}
The goal of this thesis is to study and better understand the dynamics of ellipsoidal particles in shear flow, and is a continuation of two previous masters theses \cite{AntonThesis, JonasThesis}. This is done by analysing micrometer length glass particles in a shear flow and comparing this to theoretical models. Before discussing either of these processes more in depth some background and basic theory is needed.

\subsection{Background}
The study of particle dynamics in flow began with Einstein's paper from 1905 \cite{Einstein} in which he showed that the viscocity a fluid with suspended spherical particles would increase the viscosity of the fluid and by how much. Jeffrey in his 1922 paper \cite{Jeffrey} extended these results to ellipsoidal axis-symmetrical ellipsoidal  particles as well as derived equations for the orientational dynamics of the particles, in other words how the particles would rotate as a function of time. For systems where inertial effects could be discarded the motion was found to be periodic and depending only on the initial condition of the particle. 

Developments on triaxial particles, started with Gierszewski \& Chaffey (1978)\cite{Chaffey} and was continued by Hinch \& Leal (1979)\cite{Leal} and more recently by Yarin et al in 1993\cite{Yarin}. 
The dynamics found for axis-symmetric particles by Jeffrey were periodic, but it was shown by Hinch \& Leal that some orbits would be doubly periodic, in other words following two separate independent periods. 
Yarin then was able to use numerical simulations to generate a surface of section \cite{SurfaceOfSection} for some asymmetric relations between the two minor axes showing that not only was there double periodic or quasi periodic orbits but for sufficiently large asymmetries in axes there would be chaotic orbits. 
Several other surfaces of were also produced by Johansson (2012)\cite{AntonThesis} using the same method as Yarin with higher resolution thanks to improvements in computing power. It was there shown that only very small asymmetries on the order of 1\% will lead to quasi-periodic motion for some initial conditions.

Experimental studies of these theoretical results were first done by Goldsmith and Mason in 1962\cite{Mason} who confirmed that the rotation rate matched well with that predicted from Jeffery orbits but he did not study the actual orbits. Since then most experimental research, such as by Harlen and Koch\cite{fibersspension} has been focused on diluted suspensions of particles and the increased viscocity and other properties this causes. Only tangential efforts such as by Tolga\cite{Tolga} were concerned with the Jeffery orbits. A good summary of both theoretical and experimental results was written by Petrie in 1999 \cite{Petrie}.

The first dedicated experimental work to try to measure the actual Jeffery orbits and study the orientational dynamics was done by Einarsson et al \cite{JonasExperiment} in 2011. Although there was some promising results, the vast majority of particles were asymmetric to the point of chaotic or highly quasi-periodic. Moreover the width and length of particles both varied great and could not be measured with good accuracy, meaning that the aspect ratio could not accurately be determined. This meant that although the orbits could be qualitatively shown to be similar to some Jeffery orbits no one particle could exhibit both quasi periodic and periodic motion, and there was no particle that could be well matched to a particular orbit. 

% I really want a cite for thus but how could I possibly do that.
%As no one has clearly been able to experimentally verify the results of \cite{Yarin, Leal} doing so would be very important to actually motivating using these results as well as possibly finding the limitations of this theory in real world applications. Understanding the dynamics of ellipsoidal particles in shear flow could be useful for example in our understanding of microscopic bacteria in blood stream 