\section{Introduction}
My goal in this thesis is to study and better understand the dynamics of ellipsoidal particles in shear flows. The thesis is a continuation of two previous MSc theses\cite{AntonThesis, JonasThesis}. The methodology is to experimentally measure the orientational dynamics of micrometer length glass particles in a shear flow and comparing the results to those of theoretical models. In the first part of the thesis I describe the improvements that were made to the experimental setup, most importantly an automated tracking. In the second part of the thesis the measurements and their analysis is discussed. But before discussing either of these subjects more in depth some background and theory is needed.

\subsection{Background}
Understanding the orientational dynamics of particles in flow might appear somewhat esoteric to someone unfamiliar with the field, but there is a number of topics where it is very useful. In medical applications understanding the dynamics of ellipsoidal particles such as bacteria can be relevant to a detailed understanding of their interactions with cells and other bodies. This is discussed by Tolga \emph{et al}~\cite{Tolga}. 

One of the most influential papers in the study of particle dynamics in flow was by Einstein in 1905~\cite{Einstein}. He showed how much suspended spherical particles would increase the viscosity of a fluid. Jeffery in his 1922 paper~\cite{Jeffery} extended these results to ellipsoidal  particles and derived equations for the orientational dynamics of axisymmetric particles, in other words how the particles would rotate as a function of time. For systems where inertial effects could be disregarded the motion was found to be periodic and depending only on the initial condition of the particle. 

Investigation of triaxial particles was started by Gierszewski \& Chaffey~\cite{Chaffey} and was continued by Hinch \& Leal~\cite{Leal} and more recently by Yarin \emph{et al}~\cite{Yarin}. 
The dynamics Jeffery had found for axisymmetric particles were periodic, but it was shown by Hinch \& Leal that for triaxial particles some orbits would be doubly periodic, in other words following two separate independent periods. This behaviour will in this thesis be referred to as \emph{quasi-periodic}.

Yarin \emph{et al} used numerical simulations to generate a surface of section~\cite{SurfaceOfSection} for ellipsoidal particles with different shapes. They showed that not only were there double periodic or quasi periodic orbits but when the particles were sufficiently different from axisymmetric there would be chaotic orbits. 
Several other surfaces of section were produced by Johansson ~\cite{AntonThesis} using the same method as Yarin. It was shown that even small asymmetries of the order of 1\% lead to quasi-periodic motion for some initial conditions.

Attempts to experimentally verify these theoretical results were initially performed by Goldsmith and Mason in 1962~\cite{Mason} who used flow in a glass pipe to observe the rotation rate for several different particle shapes. They confirmed that the rotation rate matched well with that predicted from Jeffery orbits but they did not study the actual orbits. Since then most experimental research, such that as by Harlen and Koch~\cite{fibersspension} has focused on how diluted suspensions of particles affect the properties of a liquid. Only tangential efforts such as by Tolga~\cite{Tolga} were concerned with the Jeffery orbits. A good summary of both theoretical and experimental results was written by Petrie~\cite{Petrie} in 1999.

The first dedicated experiments to measure the actual Jeffery orbits in angular components and verify the orientational dynamics were performed by Einarsson \emph{et al}~\cite{JonasExperiment}. Although there were some promising results, the vast majority of particles were asymmetric to the degree that their orbits were chaotic or highly quasi-periodic. Moreover the width and length of particles varied greatly and could not be measured accurately.
This meant that although the orbits could be qualitatively shown to be similar to some 
Jeffery orbits, no particular particle could be shown to exhibit both quasi periodic and periodic motion. No particle could also be well matched to a particular orbit, but different particles could be shown to be qualitatively simililar to different types of orbits. There was also few particles that very closely retraced its trajectory along the entire length of the channel when the flow was reversed, which indicates that their deviation from periodic motion was not caused by noise, as that would not be reversible.


% I really want a cite for thus but how could I possibly do that.
The goal of this thesis is to experimentally verify the results of Yarin and Hinch, Leal\cite{Yarin, Leal} and show that the same particle will show different types of motion for different initial conditions. Furthermore that different particles will show different motion for the same initial conditions based on the asymmetry. This is done by observing the orientation of a micrometer length particle in a creeping shear flow. The flow is shown to be creeping by demonstrating that the particle dynamics revert as the flow is reverted. The results are then will then be compared to theoretical predictions for different initial conditions and asymmetries.

%doing so would be very important to actually motivating using these results as well as possibly finding the limitations of this theory in real world applications. Understanding the dynamics of ellipsoidal particles in shear flow could be useful for example in our understanding of microscopic bacteria in blood stream 