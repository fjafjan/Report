The primary goal of this thesis is to experimentally verify that asymmetric particles in shear flow exhibit different types of motion for different initial conditions as was predicted by Yarin \emph{et al.}~\cite{Yarin}. To verify that the there was no significant amount of noise we look at the reversal of the flow and if the particle retraces its motion it confirms a lack of noise as well as a lack of inertial effects. To estimate the asymmetry $\epsilon$ of the particles we try to match the experimental trajectories over several of several measurements with theoretical orbits. We also utilize the winding number of quasi-periodic orbits where we can detect it to confirm that the match is valid.

Looking at Figures \ref{fig:October1Particle4_runs3and5Orbits}, \ref{fig:October1Particle4runs2and2Orbits} and \ref{fig:particleAOrbitFit} we find both quasi-periodic and periodic motion as well as all three types of orbits discussed in section \ref{sec:winding}. The winding numbers for the orbits where we can measure it are within $50\%$ of our theoretical predictions. This suggests that we have observed both quasi-periodic orbits and periodic orbits for the same particle two times. We do however have to consider the assumption we make regarding the symmetry difference between the damaged cylinders used experimentally and the flattened spheroids use in theoretical models.

This is however the results from a few measurements of two particles, there are many other measurements that have reversals where there are large differences before and after such as in Figure \ref{fig:particleABadReversal} and Figure \ref{fig:particleA5}. There are four major problems in the data 

\begin{enumerate}
\item Sinking
\item Change in orbit after a reversal
\item Too few flips to clearly estimate winding number
\item Unexplained changes in orbit 
\end{enumerate}

\section{Sinking}
One of the major problems with this setup compared to the previous setup used by Einarsson \emph{et al.} \cite{JonasExperiment} is matching the density of the fluid to that of the particles. Even with a small mismatch of $\unit[0.05]{g/ml}$ we find using eq. (\ref{eq:fallingSphere}) a sinking speed of $= \unit[4.9\cdot10^3 ]{\mu m/s}$. In earlier measurements the pump speed was $\unit[3]{\mu l/minute}$ and the particle were 60\% larger, which meant the sinking occurred more than twice as long and twice as fast which meant it was a larger problem. Even now though it can be noticeably as in longer measurements such as in Figures \ref{fig:particleB2} and \ref{fig:particleB2sinking}.

\section{Reversals where the orbit changes}
Almost every measurement with several stretches have a reversal where the particle noticeably changes orbit. This can been seen in Figures \ref{fig:particleA5}, \ref{fig:particleA4} and \ref{fig:particleB1}. We refer to such reversals as \emph{bad reversals} as they indicate that the preconditions for time reversibility are not met. While there is a trend that bad reversals occur at he end further from the pump, there are many exceptions to this. There are many cases where the orbit begins to change just as the flow is starting to reverse, such as in Figure \ref{fig:particleA4}. The particle Reynolds number depends on the velocity relative to the particle so a possible culprit could be that reversals occur too rapidly and increase the $Re_p$ such that the $Re_p << 1$ condition from Jeffery \cite{Jeffery} does not hold. We have not been able to draw any clear conclusion and solving issues with reversals would be a tremendous improvement.


\subsection{Possible expansion of the channel}
A possible cause of reversals where the people does not retrace its dynamics when the flow is reversed is that the reversals occur too quickly. A fast reversal could increase the $\Re$ so that we no longer have Stokes flow. To prevent this the reversals are incremental as discussed in Section \ref{sec:exp_setup}. 

Making the reversals slower could solve this issue, but if we look at the speed of the particle A in Figure \ref{fig:particleAspeed} and particle B in Figure \ref{fig:particleB1speed} we see that after a rather 
sharp decline in speed when the reversal starts, the acceleration is very slow. Almost all of this acceleration occurs while the pump is 
infusing or withdrawing at a fixed rate. The liquid in the channel accelerating while the pump rate is constant implies that there 
is a noticeable expansion in the channel. To verify this we look again at Figures
\ref{fig:particleAspeed} and \ref{fig:particleB1speed}.

If we have an expanded/contracted channel it should have different effects on different sides of the channel. When the flow is reversed with the particle on the side closer to the pump the impact of the channel should be limited as the only pressure felt by the particle would be built up pressure from the expansion/contraction of the channel. This must be much smaller than the pressure change from the pump. So we expect on this side that the particle would revert quickly.

However when we the flow is reversed and the particle is on opposite side of the channel from the pump things are different. Most of the channel feels the pressure difference before the particle so any excess liquid in an expanded channel will slow the reversal of the particle. This would cause a delay where there is no acceleration for some time while after the pump reversed. 

This behaviour is exactly what we see in all speed plots like \ref{fig:particleAspeed} and \ref{fig:particleB1speed}. 
For reversals on the far end of the channel there is a second dip where the particle goes to $\left|v\right|=0$, and in general reversals on the far end are noticeably slow. On the end closer to the pump this 'double dip' never occurs and accelerations are faster, but still slower than if there was no elasticity in the system.

An alternative theory for the delay in reversals would be an offset in the pump, a distance between the syringe handle and the pump holder which would need to be traversed before pumping actually begins, but this would occur at both ends of the channel equally. This does also not explain the very long acceleration of the particle when the injection rate is constant.

How this impacts the dynamics is not known, however a majority of reversals where the particle does not retrace its orbit occur at the end of the channel close to the pump, which suggests that this extra elasticity in the system might in fact be positive. 

\section{Winding number matching}
Using the score function $\hat{S}$ as described in section \ref{sec:matchorbit} to find the closest matching orbit gives us an estimate of the asymmetry and the orbit of the particle. But it only gives the best fit from a number of different orbits and asymmetries it does not say .  Instead we use the winding number to validate or dismiss a matched orbit and an estimated $\epsilon$. The winding number can only be used for the orbits where $n_z$ changes noticeably, i.e. the quasi-periodic orbits, but these are also the ones of primary interest. 
Figure \ref{fig:windingdifferent} shows that the difference in winding number for the 
same $\theta$ is on the order of a factor 2 between $\epsilon = 0.01$ and $\epsilon = 0.05$ for circular orbits, and 
still quite noticeably different between $\epsilon = 0.05$ and $\epsilon = 0.10$. However the largest difference is where the change from 
circular to bent orbit occurs. 

When we look instead at orbits for large $\left| n_z \right|$ such as in Figure 
\ref{fig:October1Particle4runs2and2Orbits} or for $n_z \approx \psi \approx 0$ such as orbit B in Figure 
\ref{fig:particleAOrbitFit} the orbits for different $\epsilon, \lambda$ 
and $i$ are all largely the same. The differences in $n_z$ are too small for us to reliably detect. This creates a problem for 
detecting particles with very small $\epsilon$. For $n_z$ that are very small, we cannot distinguish the orbits for a 
small $\epsilon$ particle with higher $\psi$ orbit for a high $\epsilon$ particle with a low $\psi$ orbit. For higher 
$n_z$ we cannot distinguish straight lines from straight lines. And in the intermediary we are unable to detect a $w 
> 20$, at best finding a sloping $n_z$ which might just be undesired reversals. Particle A has several orbits that 
are matched in the intermediary circular $n_z$ region which we can distinguish from $\epsilon = 0$ but we can not 
estimate the winding number especially well as we barely have a half period. 


\section{Unexplained behaviours}
In a number of measurements there are changes in orbit for which we have no explanation. For example in Figure \ref{fig:particleA5} the second reversal is completely sharp, the orbit virtually instantly changes, completely 'forgetting' the previous orbit. Why does this occur with the same particle, the same setup, that produce the excellent reversals in Figure \ref{fig:particleA1}. All conditions we can measure or control are the same and yet the outcome is very different. The only difference is the z coordinate, yet Figure \ref{fig:particleA3} was measured at a similar z and showed very few odd behaviours. 

\section{Width compensation}
The width compensation discussed in Sectinon \ref{sec:width_compensation} does solve the issues the previous algorithm had with 'thick' particles for $n_z$ close to 0. It does still produce small errors for $n_z \neq 0$. However it was discovered that Eq \ref{eq:widthcomp} can be solved explicitly. Given the actual length L of the particle the Euler angle $\theta$ can be solved for at each point after a measurement is complete. This would improve the resolution of peaks and allow for more accurate results.
