The primary goal of this thesis is to experimentally verify that asymmetric particles in shear flow exhibit different types of motion for different initial conditions, as was predicted by Yarin \emph{et al.}~\cite{Yarin}. To verify that there was no significant amount of noise we look at the reversal of the flow. If the particle retraces its motion it confirms a lack of noise as well as a lack of inertial effects. To estimate the asymmetry $\epsilon$ of the particles we try to match the experimental trajectories over several measurements with theoretical orbits. We also utilize the winding number of quasi-periodic orbits where it can be detected to confirm that the match is valid.

Looking at Figures \ref{fig:October1Particle4_runs3and5Orbits}, \ref{fig:October1Particle4runs2and2Orbits} and \ref{fig:particleAOrbitFit} we find both quasi-periodic and periodic motion as well as all three types of orbits discussed in section \ref{sec:winding}. The winding numbers for the orbits where we can measure it are within $50\%$ of our theoretical predictions. This suggests that we have observed both quasi-periodic orbits and periodic orbits for the same particle two times. We do however have to consider the assumption we make regarding the symmetry difference between the damaged cylinders used experimentally and the flattened spheroids use in theoretical models.

Furthermore our goal was to see that different particles will exhibit different behaviour from the same initial condition based on their asymmetry. Particle A and B exhibit quite different behaviour after the same initial condition $n_z = 0$ with a noticeably lower winding number for particle B and larger fluctuations. However with only 2 particles to compare this and only 1 good measurement for particle A it is still uncertain. 

We have not been able to show a particle exhibiting chaotic motion. This might be due to the relatively small sample of good data or due to the particles being too symmetric. Based on the Poincaré maps in Johansson~\cite{AntonThesis} we see that a larger region of chaotic orbits only appear when $\epsilon \geq 0.2$ and it seems possible based on Figure \ref{fig:particlepictures} and Figure \ref{fig:particlepictures2} that the asymmetry is lower than this for almost all particles. But is unclear if we can make that comparison between the asymmetric cylinders and the triaxial particles used in the simulations. 

These results are from a few measurements of two particles, there are many other measurements and other particles that have reversals where there are large differences before and after such as in Figure \ref{fig:particleABadReversal} and Figure \ref{fig:particleA5}. There are four major problems in the data 

\begin{enumerate}
\item Sinking
\item Change in orbit after a reversal
\item Too few flips to clearly estimate winding number
\item Unexplained changes in orbit 
\end{enumerate}

\section{Sinking}
One of the major problems with this setup compared to the previous setup used by Einarsson \emph{et al.} \cite{JonasExperiment} is matching the density of the fluid to that of the particles. Even with a small mismatch of $\unit[0.05]{g/ml}$ we find using eq. (\ref{eq:fallingSphere}) a sinking speed of $= \unit[4.9\cdot10^3 ]{\mu m/s}$. In earlier measurements the pump speed was $\unit[3]{\mu l/minute}$ and the particle were 60\% larger, which meant the sinking occurred during more than twice as long time because of the slower speed, and twice as fast due to the larger particles. This means it is no longer as big of a problem as during earlier measurement but it can still be noticeable in longer measurements such as in Figures \ref{fig:particleB2} and \ref{fig:particleB2sinking}. 

\section{Reversals where the orbit changes}
Almost every measurement with several stretches have a reversal where the particle noticeably changes orbit. This can been seen in Figures \ref{fig:particleA5}, \ref{fig:particleA4} and \ref{fig:particleB1}. While there is a trend that reversals that do not retrace their orbit occur at he end further from the pump, there are many exceptions to this. There are many cases where the orbit begins to change just as the flow is starting to reverse, such as in Figure \ref{fig:particleA4}. The particle Reynolds number depends on the velocity relative to the particle so a possible culprit could be that reversals occur too rapidly and increase the $Re_p$ such that the $Re_p << 1$ condition from Jeffery \cite{Jeffery} does not hold. We have not been able to draw any clear conclusion and solving issues with reversals would be a tremendous improvement.


\subsection{Possible expansion of the channel}
A possible cause of reversals where the people does not retrace its dynamics when the flow is reversed is that the reversals occur too quickly. A fast reversal could increase the $\Re$ so that we no longer have Stokes flow. To prevent this the reversals are incremental as discussed in Section \ref{sec:exp_setup}. 

Making the reversals slower could solve this issue, but if we look at the speed of the particle A in Figure \ref{fig:particleAspeed} and particle B in Figure \ref{fig:particleB1speed} we see that after a rather 
sharp decline in speed when the reversal starts, the acceleration is slow. Almost all of this acceleration occurs while the pump is 
infusing or withdrawing at a fixed rate. The liquid in the channel accelerating while the pump rate is constant implies that there 
is a noticeable expansion in the channel. To verify this we look again at Figures
\ref{fig:particleAspeed} and \ref{fig:particleB1speed}.

If we have an expanded/contracted channel it has different effects on different sides of the channel. When the flow is reversed with the particle on the side closer to the pump the impact of the channel is limited as the only pressure felt by the particle is the built up pressure from the expansion/contraction of the channel. Assuming that the expansion is modest we expect this built up pressure from the channel to be significantly smaller than the one exerted by the pump. So we expect that on this side of the channel the particle reverts quickly.

However when the flow is reversed and the particle is on opposite side of the channel from the pump things are different. Most of the channel feels the pressure difference before the particle so any excess liquid in an expanded channel slows the reversal of the particle. This causes a delay where there is no acceleration for some time while after the pump reversed. 

This behaviour is exactly what we see in all speed plots like \ref{fig:particleAspeed} and \ref{fig:particleB1speed}. 
For reversals on the far end of the channel there is a second dip where the particle goes to $\left|v\right|=0$, and in general reversals on the far end are noticeably slow. On the end closer to the pump this 'double dip' never occurs and accelerations are faster, but still slower than if there was no elasticity in the system.

An alternative theory for the delay in reversals is that there is an offset in the pump, a distance between the syringe handle and the pump holder which needs to be traversed before pumping actually begins. However if this was the case we would expect this delay to occur at both ends of the channel equally, and this is not what is observed. An offset in the pump does also fail to explain the long period of acceleration for the particle when the injection rate is constant.

We do not know how the slow acceleration and slow reversals impacts the dynamics of the particles, however a majority of reversals where the particle does not retrace its orbit occur at the end of the channel close to the pump. This suggests that this extra 'elasticity' in the system might in fact improve the measurements. 

\section{Winding number matching}
Using the score function $\hat{S}$ as described in section \ref{sec:matchorbit} to find the closest matching orbit gives us an estimate of the asymmetry and the orbit of the particle. But it only gives the best fit from a number of different orbits and asymmetries it does not guarantee that it is the only orbit for which there is a good match.  To try and validate this we use the winding number to validate or dismiss a matched orbit and the estimated $\epsilon$. The winding number can only be used for the orbits where $n_z$ changes noticeably, i.e. the quasi-periodic orbits, but these are also the ones of primary interest. 
Figure \ref{fig:windingdifferent} shows that the difference in winding number for the same $\theta$ is on the order of a factor 2 between $\epsilon = 0.01$ and $\epsilon = 0.05$ for sign-changing orbits, and  still quite noticeably different between $\epsilon = 0.05$ and $\epsilon = 0.10$. However the largest difference between different $\epsilon$ is where the change from sign-changing to sign-preserving orbits occur. 

For all the measurements where we can estimate the winding number we find a value less than a factor 2 from the one of the theoretically matched orbit. For particle A the ratio between the estimated winding number and the theoretical winding number $\frac{w_{exp}}{w_{theory}}$is ${\sim}0.8$. For particle B we find similar with differences varying between $0.8$ and $1.2$ for measurements 1 and 2 and $0.75$ and $1.4$ for measurement 3. This suggests that the $\epsilon$ values are not exact, but in general the winding numbers are close to what is predicted and thus the matching orbit and $\epsilon$ is close to the actual one. 

When we look instead at orbits for large $\left| n_z \right|$ such as in Figure 
\ref{fig:October1Particle4runs2and2Orbits} or for $n_z \approx \psi \approx 0$ such as orbit B in Figure 
\ref{fig:particleAOrbitFit} the orbits for different $\epsilon, \lambda$ 
and $i$ are all largely the same. The differences in $n_z$ are too small for us to reliably detect. This creates a problem for 
detecting particles with very small $\epsilon$. For $n_z$ that are very small, we cannot distinguish the orbits for a 
small $\epsilon$ particle with higher $\psi$ orbit for a high $\epsilon$ particle with a low $\psi$ orbit. For higher 
$n_z$ we cannot distinguish straight lines from straight lines. And in the intermediary we are unable to detect a $w 
> 20$, at best finding a sloping $n_z$ which might just be undesired reversals. Particle A has several orbits that 
are matched in the intermediary sign-changing $n_z$ region which we can distinguish from $\epsilon = 0$ but we can not 
estimate the winding number especially well as this measurement is barely half period of the longer period $\theta_1$. 


\section{Unexplained behaviours}
In a number of measurements there are changes in orbit for which we have no explanation. For example in Figure \ref{fig:particleA5} the second reversal is completely sharp, the orbit virtually instantly changes, completely 'forgetting' the previous orbit. Why does this occur with the same particle, the same setup, that produce the excellent reversals in Figure \ref{fig:particleA1}. All conditions we can measure or control are the same and yet the outcome is very different. The only difference is the z coordinate, yet Figure \ref{fig:particleA3} was measured at a similar z and showed very few odd behaviours. 

\section{Width compensation}
The width compensation discussed in Sectinon \ref{sec:width_compensation} does solve the issues the previous algorithm had with 'thick' particles for $n_z$ close to 0. It does still produce small errors for $n_z \neq 0$. However it was discovered that Eq \ref{eq:widthcomp} can be solved explicitly. Given the actual length L of the particle the Euler angle $\theta$ can be solved for at each point after a measurement is complete. If this was implemented it would improve the resolution of peaks and allow for more accurate results.
