Looking at figures \ref{fig:October1Particle4_runs3and5Orbits}, \ref{fig:October1Particle4runs2and2Orbits} and \ref{fig:particleAOrbitFit} we find all three types of orbits discussed in section \ref{sec:winding} and their winding numbers for the orbits where we can measure it also agree well with our theoretical predictions. This is however only for the stretches that do work well, and even in those measurements there are many reversals where there are large differences before and after such as in figure \ref{fig:particleABadReversal}. 

In general orbits with higher $n_z$ have very periodic orbits, whereas low $n_z$ do not. 

We see four major problems in the data 

\begin{enumerate}
\item Sinking
\item Bad reversals
\item Too few flips to clearly estimate winding number
\item Unexplained changes in orbit
\end{enumerate}

\section{Sinking}
One of the major problems with this setup compared to the previous setup is the density matching, given a density mismatch of $\unit[0.05]{g/ml}$ we find using \ref{eq:fallingSphere} a falling speed of $\frac{2}{9} \frac{0.05 \cdot 9.82 \cdot (10^{-5})^2}{2.4\cdot 10^-3} = \unit[4.9\cdot10^3 ]{\mu m/s}$. In earlier measurements the pump speed was $\unit[3]{\mu l/minute}$ and the particle were 60\% larger, which meant the sinking occurred more than twice as long and twice as fast which meant it was a larger problem. Even now though it can be noticeably as in longer measurements such as in figures \ref{fig:particleB2} and \ref{fig:particleB2sinking}.

\section{Reversals}
Almost every measurement with several stretches will have on reversal where the particle noticeably changes orbit. This can been seen figures \ref{fig:particleA5}, \ref{fig:particleA4} and \ref{fig:particleB1}. While there is a trend that bad reversals occur at he end further from the channel there are many exceptions to this. There are many cases where the orbit begins to change just as the flow is starting to reverse, such as in figure \ref{fig:particleA4} so a possible culprit would then be that reversals occur too rapidly.



\subsection{Velocity behaviour}
A possible cause of bad reversals are too rapid reversals and to prevent this the reversals are staggered. At the start of a reversal the infusion/withdrawal rate is reduced to 50\% for 10 seconds, then stopped completely for 10 seconds and reverted at 50\% for another 10 seconds before resuming at full speed.

Possibly increasing this staggering on the first part might make a difference, but if we look at the plot of the 
speed of the particle in figure \ref{fig:particleAspeed} and \ref{fig:particleB1speed} we see that after a rather 
sharp decline in speed the acceleration is very slow. Almost all of this acceleration occurs while the pump is 
infusing or withdrawing at a fixed rate. Now the liquid accelerating while the pump rate constant suggests that there 
is a noticeably expansion in the channel. To verify this we can look more closely at he speed graphs 
\ref{fig:particleAspeed} and \ref{fig:particleB1speed}.

If we have an expanded/contracted channel it should have different effects on different sides of the channel. We 
expect that on the side closer to the pump the expanding channel will simply absorb part of the fluid infused causing 
a slowed acceleration but it would start more or less right as the pump starts infusing/withdrawing. Meanwhile on the 
far end of the channel the 'extra' fluid in the channel could be withdrawn before any pressure is felt on the far 
side. This would cause a delay where there is no acceleration for some while after the pump started acceleration. 

This behaviour is exactly what we see in all speed plots like \ref{fig:particleAspeed} and \ref{fig:particleB1speed}, 
on reversals on the far end there is a second dip where the particle starts at $\left|v\right|=0$ for some extra 
seconds. On the end closer to the pump this never occurs. 

An earlier theory for the delay would be an offset in the pump, for example a distance between the syringe handle and the pump holder which would need to be traversed, but this would occur at both ends of the channel and would not explain the very long acceleration of the fluid. 

\section{Winding number matching}
While using the score function $\hat{S}$ to find the closest matching orbit is useful, it only gives the best fit and 
does not actually show that the orbit is close (just more close than the other ones). Instead the best tool for 
validating, or dismissing, a matched orbit and an estimated $\epsilon$ is the winding number for the orbits where 
this can be done. If we look at figure \ref{fig:windingdifferent} we see that the difference in winding number of the 
same $\theta$ is on the order of a factor 2 between $\epsilon = 0.01$ and $\epsilon = 0.05$ for circular orbits, and 
still quite noticeably different between $\epsilon = 0.05$ and $\epsilon = 0.10$, especially where the change from 
circular to bent orbit occurs. 

When we look instead at orbits for large $\left| n_z \right|$ like in figure 
\ref{fig:October1Particle4runs2and2Orbits} or for $n_z \approx \psi \approx 0$ like orbit B in figure 
\ref{fig:particleAOrbitFit} there is not much information to extract. The orbits for different $\epsilon, \lambda$ 
and $i$ all largely the same, the differences are too small for us to reliably detect. This creates a problem for 
detecting particles with very small $\epsilon$. For $n_z$ that are very small, we cannot distinguish the orbits for a 
small $\epsilon$ particle with higher $\psi$ orbit for a high $\epsilon$ particle with a low $\psi$ orbit. For higher 
$n_z$ we cannot distinguish straight lines from straight lines. And in the intermediary we are unable to detect a $w 
> 20$, at best finding a sloping $n_z$ which might just be undesired reversals. Particle A has several orbits that 
are matched in the intermediary circular $n_z$ region which we can distinguish from $\epsilon = 0$ but we can not 
estimate the winding number especially well as we barely have a half period. If we indeed had a circular orbit with 
$w = 19.5$ as predicted we need to use the end point 


\section{Unexplained behaviours}
In a number of measurements there are changes in orbit for which we have no good explanation. For example in figure \ref{fig:particleA5} the second reversal is cmpletely sharp, the orbit virtually instantly changes, completely 'forgetting' the previous orbit. Why does this occur with the same particle, the same setup, seemingly the same conditions that produce the excellent reversals in figure \ref{fig:particleA1}. The only difference is the z coordinate, yet figure \ref{fig:particleA3} was measured at a similar z and showed very few odd behaviours. This could be explained

