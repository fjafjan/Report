The automated tracking discussed in Section \ref{sec:tracking} encountered problems as the experimental setup was changed. When implementing the optical tweezers the particles used were changed from 5$\mu$m width to 3$\mu$m width and the objective used was changed to a 60x objective. This meant that the margins to track were made smaller as the particle appears larger in the image. 

In order to reduce sinking the speed of the fluid was also increased by a factor 2.5. These two changes together made the tracking unable to reliable track the particle during flow reversals as the simplistic model of constant speed does not hold. A few models for constant acceleration during reversals and variations thereof were implemented but were also unreliable. As such, the automated tracking was not used during the measurements presented in Chapter \ref{chap:results}. 

Due to the increased flow speed, an automated tracking is also less necessary as measurements take approximately 60\% less time. But an automated tracking would still be a significant improvement in saving work time for whoever is doing the measurements. Future efforts could certainly go into finding a better predictive to free up more time to solve other problems instead of manually tracking the particles. If optical tweezers are used that computer controllable and density matching problems are solved, the entire measuring procedure could be automated.

