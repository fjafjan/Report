\section{LBM for the Nernst-Planck equation}\label{sec:lbm:np}
The method presented here is based on representing the Nernst-Planck
equation, eq. \eqref{eq:et:np}, as an equation of advection-diffusion
type. Considering the quantity:  

\begin{equation}\label{eq:lbm:eff_adv}
\bar{\ubf} = \ubf -
  \frac{zq_eD}{k_BT}\nabla\psirm
\end{equation}
as an effective advective velocity, we have:

\begin{equation}\label{eq:lbm:adv-dif}
\dfrac{\partial \rho}{\partial t} + \nabla \cdot ( \bar{\ubf} \rho -
  D\nabla \rho ) = 0
\end{equation}
which is a mass conservation equation with fluxes from diffusion and
from advection respectively. The letter $\C$ for denoting the charge
concentration has in this section been replaced by the letter $\rho$
to avoid the risk of confusing it with the lattice velocities
which traditionally are denoted by ${\ci}$.

A collision operator of BGK type, eq. \eqref{eq:lbm:bgk} will be used
together with a D2Q9 lattice. The lattice-Boltzmann equation then
reads:

\begin{equation}
f_i(\x + \cbf_i\delta_t, t + \delta_t) - f_i(\x, t) = -\omega \left[ f_i(\x, t) - f_i^{(eq)}(\x, t) \right]
\end{equation}
with $\{\cbf_i\}_{i=0}^{Q-1}$ for the D2Q9 lattice as in
eq. \eqref{eq:lbm:d2q9_c}. The equilibrium function, $\feq$, is chosen
as \cite{alexey-tobias}:

\begin{equation}\label{eq:lbm:np_feq}
\feq = w_i \rho \left ( 1 + \frac{\ci \cdot \ubar}{c_s^2} \right)
\end{equation}
with the weights, $w_i$, as in eq. \eqref{eq:lbm:weights}. The charge
density and charge flux density is obtained by taking the zeroth and
first moments of the distribution function respectively, i.e:

\begin{equation}\label{eq:lbm:rho_mom}
\rhorm = \sum_i \fii
\end{equation}
and
\begin{equation}\label{eq:lbm:j_mom}
\jj = \sum_i \fii \ci
\end{equation}
The diffusion constant, $D$, is related to the relaxation parameter
$\omega$ through 

\begin{equation}\label{eq:lbm:np_D}
D = c_s^2 \left( \frac{1}{\omega} - \frac{1}{2} \right).
\end{equation}

\subsection{Asymptotic analysis}\label{sec:lbm:asym_np}
To motivate the appearance of the above suggested method for solving
eq. \eqref{eq:lbm:adv-dif} and for showing under what premises the
method is valid, the macroscopic limit of the discrete scheme will now
be analysed in an asymptotic manner. Note that for the advection
diffusion equation mass is but flux is not conserved.

From the expansion of $\fii$ in eq. \eqref{eq:lbm:fi_exp} and from
eqs. \eqref{eq:lbm:rho_mom} and \eqref{eq:lbm:j_mom} follow the
expansions of the mass density and flux respectively as

\begin{equation}\label{eq:lbm:rho_exp}
\rho = \rhoe{0} + \ep\rhoe{1} + \ep^2\rhoe{2} + \ep^3\rhoe{3} + \bigO{\ep^4}
\end{equation} 
and

\begin{equation}
\jj = \je{0} + \ep\je{1} + \ep^2\je{2} + \ep^3\je{3} + \bigO{\ep^4}
\end{equation} 
The advective velocity is also expanded as: 

\begin{equation}
\ubar = \ubare{0} + \ep\ubare{1} + \ep^2\ubare{2} + \ep^3\ubare{3} + \bigO{\ep^4}
\end{equation}
By plugging these expansion into the equilibrium distribution
eq. \eqref{eq:lbm:np_feq}, the expansion in
eq. \eqref{eq:lbm:fi_eq_exp} is obtained. The terms of order zero is
used in the zeroth order equation of the LBE, eq. \eqref{eq:lbm:ep0},
which gives

\begin{equation}
\fie{0} = w_i\rhoe{0} \left( 1 + \frac{\ci \cdot \ubare{0}}{c_s^2} \right).
\end{equation}
However, since we are only considering advection in the low Mach
limit, i.e. $|\ubar| \sim \ep$, we will in this analysis assume that
$\ubare{0} = 0$. $\ubar$ will then be of order $\ep$ to leading
order. It is possible to show \cite{junk-asym} that this assumption
holds if $\ubar$ is initialised properly, i.e. small and if no major
momentum sources are present. Thus the expression for $\fie{0}$
reduces to

\begin{equation}\label{eq:lbm:np_fi0}
\fie{0} = w_i\rhoe{0}.
\end{equation}

We now continue to the equation of order one in $\ep$,
eq. \eqref{eq:lbm:ep1}. Taking the zeroth moment gives the equation $0
= 0$ which indeed is true but not very useful. Note that the right
hand side vanishes due to mass conservation. $\fie{1}$ will be needed
in the next step and is, by using eq. \eqref{eq:lbm:np_fi0}:

\begin{equation}
\fie{1} = -\frac{1}{\omega} (\pd)(w_i\rhoe{0}) + w_i\left( \rhoe{1} +
\rhoe{0} \frac{\ci \cdot \ubare{1} }{c_s^2} \right).
\end{equation}
Taking the first moment of $\fie{1}$ gives the leading order in the
flux ($\je{0} = 0$ since $\ubare{0} = 0$):

\begin{equation}
\je{1} = \rhoe{0}\ubare{1} - c_s^2/\omega\nabla\rhoe{0}
\end{equation} 

Continuing to the equation of order two in $\ep$,
eq. \eqref{eq:lbm:ep2} and taking the zeroth moment of the equation
gives

\begin{equation}
\nabla \cdot \je{1} + \partial_t \rhoe{0} + c_s^2/2\nabla^2 \rhoe{0} = 0
\end{equation}
and by inserting the expression for $\je{1}$ we end up with

\begin{equation}
\partial_t \rhoe{0} + \nabla \cdot \left [ \rhoe{0}\ubare{1} - c_s^2
  \left( \frac{1}{\omega} - \frac{1}{2} \right)\nabla\rhoe{0} \right ]
= 0
\end{equation}
which is an advection diffusion equation with a diffusion constant as
in eq. \eqref{eq:lbm:np_D}. Since $\rhoe{0}$ fulfils the equation of
interest, the solution $\rho$ that we get from the lattice-Boltzmann
method is at least of first order accuracy. To determine the exact
order of accuracy, higher order terms, $\rhoe{k > 0}$, must be
determined. If also all those terms would be zero the method would be
exact, unfortunately that is not the case. The analysis of higher order
terms will not be performed here, but it is possible to show that
$\rhoe{1}$ is zero only under certain premises, i.e. for a proper
initialisation \cite{alexey-tobias}. $\rhoe{2}$ is however in general
non-zero and the obtained solution is thus second order accurate.  
