\section{LBM for Poisson's equation}
As Poisson's equation, eq. \eqref{eq:pb}, is considered, a fundamental
difference to the Nernst-Planck and Navier-Stokes equations is
imediatley noted. It is not a differential equation of parabolic but
of elliptic type. If we consider what has been said about the LBM so
far in this chapter, it seems that it is a method that deals with
parabolic equations and not elliptic. However, by introducing a time
derivative to the Poisson's equation and only considering the steady state
solution the LBM may be used even for this equation. A diffusion-like
equation with a source term is then obtained

\begin{equation}\label{eq:lbm:poi_dif}
\partial_t\rho + \nabla^2 \rho = \R
\end{equation}
where $\R$ is the right-hand side of the Poisson's equation.

A LBM for this equation will now be formulated. The LBE to be solved
in this case is, an equation of the form:

\begin{equation}
f_i(\x + \cbf_i\delta_t, t + \delta_t) - f_i(\x, t) = -\omega \left[
  f_i(\x, t) - f_i^{(eq)}(\x, t) \right] + \Gi(\R)
\end{equation}
where $\Gi(\R)$ is an addition to the BGK collision operator to account
for the source term $\R$ in eq. \eqref{eq:lbm:poi_dif}. The equilibrium
distribution is given by

\begin{equation}
\feq = w_i \rho
\end{equation}
where $w_i$ are the weights in eq. \eqref{eq:lbm:weights}.

The quantity $\rho$ which in this work is interpreted to electric
potential in the steady state is determined by

\begin{equation}
\rho = \sum_i \fii
\end{equation}
and the quantity which, to us, is of even more interest, i.e. the
electric field, is given by

\begin{equation}
\nabla \rho = - \frac{\omega}{c_s^2 \delta_x} \sum_i \fii \ci.
\end{equation}
Finally, the addtional source term in the collision operator, $\Gi(\R)$,
is given by

\begin{equation}\label{eq:lbm:gi}
\Gi(\R) = w_ic_s^2 \left( \frac{1}{2} - \frac{1}{\omega} \right)\R.
\end{equation}

\subsection{Asymptotic analysis}\label{sec:lbm:asym_pe}
The given method for solving eq. \eqref{eq:lbm:poi_dif} will now be
motivated by preforming an asymptotic analysis of the suggested LBE.

$\fii$ is expanded as in eq. \eqref{eq:lbm:fi_exp} which gives an
expansion of $\rho$ as in eq. \eqref{eq:lbm:rho_exp}. Further, 
the source term $\R$ is expanded: 

\begin{equation}\label{eq:lbm:R_exp}
\R = \Rexp{0} + \ep\Rexp{1} + \ep^2\Rexp{2} + \ep^3\Rexp{3} + \bigO{\ep^4}
\end{equation} 
which also gives an expansion for the LBE source term as

\begin{equation}\label{eq:lbm:G_exp}
\Gi = \Gie{0} + \ep\Gie{1} + \ep^2\Gie{2} + \ep^3\Gie{3} + \bigO{\ep^4}.
\end{equation} 

The equilibrium distribution is expanded by plugging in the expansion
of $\rho$ from eq. \eqref{eq:lbm:rho_exp}. This gives for the zeroth
order equation in $\ep$, eq. \eqref{eq:lbm:ep0} that

\begin{equation}
\fie{0} = w_i\rhoe{0} + \frac{1}{\omega}\Gie{0}
\end{equation}
However, taking the first moment of the equation gives $\Gie{0} = 0$
and the above expression for $\fie{0}$ reduces to

\begin{equation}
\fie{0} = w_i\rhoe{0}.
\end{equation}

Coninuing to the equation of first order in $\ep$,
eq. \eqref{eq:lbm:ep1} and taking the zeroth moment gives that also
$\Gie{1} = 0$. Taking the first moment of the same equation  gives

\begin{equation}
\nabla \rhoe{0} = - \frac{\omega}{c_s^2} \sum_i \fie{1} \ci \approx -
\frac{\omega}{c_s^2 \delta_x} \sum_i \fii \ci
\end{equation}
Here, for the approximation, the expansion of $\fii$ is used together
with the fact that $\ep = \delta_x$ in dimensionless variables.

The next equation in $\ep$, i.e. the one of order two, will
eventually give us what we are looking for. Taking the zeroth moment
gives

\begin{equation}
-\frac{c_s^2}{\omega} \nabla^2 \rhoe{0} + \partial_t \rhoe{0} +
\frac{c_s^2}{2}\nabla^2 \rhoe{0} = \sum_i \Gie{2}(\Rexp{2})
\end{equation}
From this equation, it is deduced that in order to obtain Poisson's
equation in a steady state situation, i.e. when $\partial_t \rhoe{0} =
0$ the right-hand side must fulfill the follwong condition

\begin{equation}
\sum_i \Gie{2}(\Rexp{2}) = c_s^2\left( \frac{1}{2} - \frac{1}{\omega} \right)\Rexp{2}
\end{equation}
One such $\Gie{2}(R)$ is the one described in eq. \eqref{eq:lbm:gi}. 
