\documentclass[9pt,handout]{beamer} % mathserif for normal math fonts.

\usefonttheme[onlymath]{serif}
\usepackage{amsmath,mathtools}
\usepackage{calc}
\usepackage{caption}
\usepackage{color}

% KRJ: Removed because of xelatex
%\usepackage[EU1]{fontenc}
%\usepackage[utf8]{inputenc}
%\usepackage{lmodern}
\usepackage[swedish,english]{babel}
\usepackage[babel]{csquotes}

% KRJ: xelatex stuff
\usepackage{xltxtra}
%\usepackage{polyglossia}
\usepackage{fontspec}
%\usepackage[]{unicode-math}
%\setmainfont{XITS}
%\setmathfont{XITS Math}
%\usepackage{csquotes}

\usepackage{subcaption}
\usepackage{listings}
\lstset{ %
language=Python,     
basicstyle=\tiny,   
numbers=none,         
numberstyle=\tiny,    
stepnumber=1,         
numbersep=5pt,        
backgroundcolor=\color[gray]{0.95},  
showspaces=false,               
showstringspaces=false,       
showtabs=false,             
frame=single,        
tabsize=2,         
captionpos=b,      
breaklines=break,  
linewidth=0.98\linewidth,
xleftmargin=0.0\linewidth,
breakatwhitespace=false,   
%escapeinside={\%}{)}     
    identifierstyle=\ttfamily,
    keywordstyle=\color[rgb]{0,0,1},
    commentstyle=\color[rgb]{0.133,0.545,0.133},
    stringstyle=\color[rgb]{0.627,0.126,0.941}
}

\usetheme[titleflower=true]{chalmers} % titleflower = true or false
\title{Coupled calculations in OpenFOAM - }
\subtitle{Multiphysics handling, structures and solvers, Gothenburg Region OpenFOAM User Group Meeting } % optional
\author[Klas Jareteg]{Klas Jareteg} % [short author (optional)]{many authors}
\institute{Chalmers University of Technology}
\date{November 14, 2012}
\footer{\insertshortauthor, Gotheburg Region OpenFOAM User Group Meeting } 

\usepackage[style=verbose]{biblatex}
\addbibresource{/home/klas/Documents/PhD/Library/Sources.bib}
\let\oldfootnotesize\footnotesize
\renewcommand*{\footnotesize}{\oldfootnotesize\tiny}
\newcommand{\KRJScale}{0.33}
\newcommand{\Ukrj}{\mathbf{U}}
\newcommand{\Sf}{\mathbf{S}_\mathrm{f}}
\newcommand{\Pp}{P}
\newcommand{\D}{\overline{\mathbf{D}_\mathrm{f}}}
\newcommand{\Dnobar}{\mathbf{D}_\mathrm{f}}

\usepackage{../styles/KJCommands}

% KRJ, Colorbox for equations and theorems
%\usepackage[listings,theorems]{tcolorbox}
%\newenvironment{krjbox}[1]{
%\centering
%\tcolorbox[width=0.9\textwidth,colback=chalmersblue!5,colframe=chalmersblue,title=#1,fontupper=\scriptsize]}
%{\endtcolorbox}

\AtBeginSection[]
{
\begin{frame}
%\usebeamerfont{section title}\insertsection
\begin{columns}
\begin{column}[]{0.5\textwidth}
%\begin{center}
\vbox to .8\textheight{
\huge\bf\insertsection
\\[5cm]\vfill
}
%\end{center}
\end{column}
\begin{column}[b]{0.5\textwidth}
\vbox to .3\textheight{
\small{\tableofcontents[currentsection]}
\vfill
}
\end{column}
\end{columns}
\end{frame}
}

%\AtBeginSection[]{
%  \setbeamercolor{section in toc shaded}{use=structure,fg=structure.fg}
%  \setbeamercolor{section in toc}{fg=chalmersblue}
%  \setbeamercolor{subsection in toc shaded}{fg=black}
%  \setbeamercolor{subsection in toc}{fg=chalmersblue}
%  \frame<beamer>{
%  %\begin{multicols}{2}
%      \frametitle{Outline}
%      \setcounter{tocdepth}{1}  
%      \LARGE{
%        \tableofcontents[currentsection]
%      }
%  %\end{multicols} 
%  }
%}

% Notes in the document
%\setbeameroption{show notes} 

\newlength{\parskipbackup}
\setlength{\parskipbackup}{\parskip}
\newlength{\parindentbackup}
\setlength{\parindentbackup}{\parindent}

\let\notebackup\note
\renewcommand{\note}[1]{\notebackup{%
\mode<handout>{\addtocounter{page}{-1}}%
\setlength{\parindent}{0ex}%
\setlength{\parskip}{10pt}%
\noindent%
{\normalsize{}#1}%
\setlength{\parskip}{\parskipbackup}%
\setlength{\parindent}{\parindentbackup}%
}%
}

\begin{document}

%%%%%%%%%%%%%%%%%%%%%%%%%%%%%%%%%%%%%%%%%%%%%%%%%%%%%%%%%%%%%%%%%%%%%%%%%%%%%%%
%%%%%%%%%%%%%%%%%%%%%%%%%%%%%%%%%%%%%%%%%%%%%%%%%%%%%%%%%%%%%%%%%%%%%%%%%%%%%%%
%%%%%%%%%%%%%%%%%%%%%%%%%%%%%%%%%%%%%%%%%%%%%%%%%%%%%%%%%%%%%%%%%%%%%%%%%%%%%%%

\begin{frame}[plain]
 \titlepage
\end{frame}

\begin{frame}
\frametitle{Outline}
\begin{itemize}\itemsep10pt
\item Region coupling and block coupling
\item Overview of coupled formats in OpenFOAM
\item Application 1: Multiphysics simulations for nuclear reactors
\item Application 2: Block coupled solver for incompressible flow
\item Summary and future outlook 
\end{itemize}
\end{frame}

%%%%%%%%%%%%%%%%%%%%%%%%%%%%%%%%%%%%%%%%%%%%%%%%%%%%%%%%%%%%%%%%%%%%%%%%%%%%%%%%
%%%%%%%%%%%%%%%%%%%%%%%%%%%%%%%%%%%%%%%%%%%%%%%%%%%%%%%%%%%%%%%%%%%%%%%%%%%%%%%%
%%%%%%%%%%%%%%%%%%%%%%%%%%%%%%%%%%%%%%%%%%%%%%%%%%%%%%%%%%%%%%%%%%%%%%%%%%%%%%%%

\section{On coupled problems}
\begin{frame}%[allowframebreaks]
\setbeamercovered{transparent}
\frametitle{Coupled problems}
\begin{itemize}
    \item Multiple field problems and/or equations problems, e.g.:
    \begin{itemize}\itemsep6pt
        \item Navier-Stokes
        \item Conjugate heat-transfer
        \item Nuclear reactor modeling
        \item Multi-group radiation
        \item Multiphase modeling
    \end{itemize}
    \pause
    \item Not enough to solve one problem. 
\end{itemize}
\vfill
\pause
\bf How to solve the coupling of the fields/equations/regions?
\end{frame}

%%%%%%%%%%%%%%%%%%%%%%%%%%%%%%%%%%%%%%%%%%%%%%%%%%%%%%%%%%%%%%%%%%%%%%%%%%%%%%%%
%%%%%%%%%%%%%%%%%%%%%%%%%%%%%%%%%%%%%%%%%%%%%%%%%%%%%%%%%%%%%%%%%%%%%%%%%%%%%%%%
%%%%%%%%%%%%%%%%%%%%%%%%%%%%%%%%%%%%%%%%%%%%%%%%%%%%%%%%%%%%%%%%%%%%%%%%%%%%%%%%

\section{Region coupling}
\subsection{Background}
\begin{frame}[t]
\frametitle{Background}
\setbeamercovered{transparent}
\begin{columns}
    \begin{column}{0.55\textwidth}
    \begin{itemize}
    \item<1->{\bf Characteristic:}
    \begin{itemize}
        \item Multiple materials (solid/fluid)
        \item Similar physics in many regions ($T$)
        \item Coupled boundaries
    \end{itemize}
    \vskip10pt
    \item<2>
    {\bf Solution methods:}
    \begin{itemize}
        \item Segregated solver - Iterative solution of region by region
        \item Coupled solver - Solve multiple regions at once
    \end{itemize} 
    \end{itemize} 
    \end{column}
    \begin{column}[]{0.45\textwidth}
    \begin{figure}[ht]
        \centering
        \includegraphics[width=1.0\textwidth]{/home/klas/Documents/PhD/Courses/OpenFOAM/Projects/1/report/handedin/figures/conjugateCavity-T.pdf}
        \caption{\OF{conjugateCavity}, dual region simulation}
    \end{figure}
    \end{column}
\end{columns}
    \begin{figure}[ht]
        \centering
        \includegraphics<2>[width=0.6\textwidth]{figures/RegionCoupling.pdf}
        %\caption{}
    \end{figure}
\vfill
\end{frame}

\subsection{OpenFOAM}
\begin{frame}%[allowframebreaks]
\frametitle{Regional coupling in OpenFOAM I}
\setbeamercovered{transparent}
\begin{itemize}
    \item Regional coupling availble through \OF{coupleFvMatrix}.
            \lstinputlisting[title={Energy equation in \OF{conjugateHeatFoam}},language=C++,linerange=10-34,firstnumber=10]{/home/klas/OpenFOAM/OpenFOAM-1.6-ext/applications/solvers/coupled/conjugateHeatFoam/solveEnergy.H}
\end{itemize}
\end{frame}

\begin{frame}%[allowframebreaks]
\frametitle{Regional coupling in OpenFOAM II}
\setbeamercovered{transparent}
\begin{itemize}
    \item \OF{regionCouple} boundary condition at coupled patches
    \pause
    \item Existing solvers \OF{CG}, \OF{BiCGStab}, \OF{BiCG}, \OF{smoothSolver}
    \pause
    \item Parallelizable 
    \pause
    \item Utilities (mostly) compatible with multiple regions (\OF{splitMesh}, \OF{foamToVTK}, \OF{sample}, \OF{decomposePar}, \OF{reconstructPar}) 
\end{itemize}
\pause
\vfill

\rule{0.5\textwidth}{0.5pt}\newline
\scriptsize
{\bf Additional sources:}
\begin{itemize}
    \item \fullcite{jasak:of:w7}
    \item \fullcite{krj:master}
\end{itemize}
\end{frame}

\subsection{Application: multiphysics in nuclear reactors}

\begin{frame}%[allowframebreaks]
\setbeamercovered{transparent}
\frametitle{Application: multiphysics in nuclear reactors}
Nuclear reactor cores: multiphysics environment
\begin{figure}[ht]
    \centering
    \includegraphics[scale=0.55]{/home/klas/Documents/Master/Reports/siamuf/figures/Multi5.pdf}
\end{figure}
\end{frame}

\begin{frame}%[allowframebreaks]
\setbeamercovered{transparent}
\begin{columns}
    \begin{column}[]{0.5\textwidth}
    \begin{itemize}\itemsep8pt
        \item \textbf{Multi-region environment}: Both solid and fluid, different types of materials
        \item<2> \textbf{Fields in all regions}:
        \begin{itemize}\itemsep8pt
            \item Temperature (solid: conduction, fluid: convection)
            \item Multi-group neutron density field (same model equation all regions)
        \end{itemize}
    \end{itemize}
    
    \end{column}
    \begin{column}[]{0.5\textwidth}
    \begin{figure}[ht]
        \centering
        \includegraphics[width=1.0\textwidth]{figures/Fuel.pdf}
        \caption{Slice of nuclear fuel pin}
    \end{figure}    
    \end{column}
\end{columns}
\end{frame}

\begin{frame}%[allowframebreaks]
\frametitle{OpenFOAM implementation}
\setbeamercovered{transparent}
\begin{itemize}\itemsep8pt
    \item Implemented using OpenFOAM 1.6-ext.
    \item Region coupling for an arbitrary number of pins
    \item Segregated solver considering field dependencies
    %\item Eigenvalue solver (power iteration method) using the neutron density fields
\end{itemize}

\vfill
\pause
\rule{0.5\textwidth}{0.5pt}\newline
\scriptsize
{\bf Details:}
\begin{itemize}
    \item \fullcite{krj:master}
    \item \fullcite{krj:website}
\end{itemize}
\end{frame}

\begin{frame}%[allowframebreaks]
\frametitle{Example results}
\vspace{16pt}
\begin{figure}[!htb]
\centering
\begin{subfigure}{0.46\textwidth}
\centering
\includegraphics[width=1.0\textwidth]{/home/klas/Documents/PhD/Articles/2-ForMoC2013/figures/results/TAll.pdf}
\caption{All regions}
\end{subfigure}
\begin{subfigure}{0.46\textwidth}
\centering
\includegraphics[width=1.0\textwidth]{/home/klas/Documents/PhD/Articles/2-ForMoC2013/figures/results/T.pdf}
\caption{Moderator only}
\end{subfigure}
\caption{\bf Radial temperature profiles.} 
%\label{fig:res:T:radial}
\end{figure}
\end{frame}

%\begin{frame}%[allowframebreaks]
%\frametitle{Learning outcomes}
%\begin{itemize}
%    \item Region coupling in OpenFOAM 1.6-ext functional for large number of regions
%    \item Not all solvers applicable for all problems
%    \item Parallelization working, but some caution is needed
%    \item OpenFOAM 1.6-ext is well suited for multiphysics problems
%    \item Many regions gives $\rightarrow$ lots of data (initial/boundary conditions). Need scripting.
%\end{itemize}
%\end{frame}


%%%%%%%%%%%%%%%%%%%%%%%%%%%%%%%%%%%%%%%%%%%%%%%%%%%%%%%%%%%%%%%%%%%%%%%%%%%%%%%%
%%%%%%%%%%%%%%%%%%%%%%%%%%%%%%%%%%%%%%%%%%%%%%%%%%%%%%%%%%%%%%%%%%%%%%%%%%%%%%%%
%%%%%%%%%%%%%%%%%%%%%%%%%%%%%%%%%%%%%%%%%%%%%%%%%%%%%%%%%%%%%%%%%%%%%%%%%%%%%%%%


\section{Field coupling}

\subsection{Background}
\begin{frame}%[allowframebreaks]
\setbeamercovered{transparent}
\frametitle{Coupling 2: Field coupling}
\begin{columns}
    \begin{column}[]{0.6\textwidth}
    \begin{itemize}
    \item {\bf Characteristic:}
    \begin{itemize}
        \item Multiple fields ($\alpha, U_g, U_l, p$)
        \item Couplings between fields
        \item Same mesh (typically)
        \item Field dependent parameters
    \end{itemize}
    \item<2>{\bf Solution methods:}
    \begin{itemize}
        \item Segregated solver - Iterative solution. Solve one field at a time. Explicit coupling.
        \item Coupled solver - Solve all/multiple fields at once. Implicit coupling.
    \end{itemize}
    \end{itemize}
    \end{column}
    \begin{column}[]{0.4\textwidth}
    \begin{figure}[ht]
        \centering
        \includegraphics[width=1.1\textwidth]{/home/klas/Documents/PhD/Courses/OpenFOAM/Projects/1/report/handedin/figures/VoidStream.pdf}
        \caption{\OF{bubbleColumn}}
    \end{figure}
    \end{column}
\end{columns}
\pause
\vfill
\end{frame}

%\begin{frame}%[allowframebreaks]
%\frametitle{Coupled approach 1: extended system matrix}
%\begin{itemize}
%\item Consider coupled system:
%\begin{eqnarray}
%\mathbf{A}(y)x=a\\
%\mathbf{B}(x)y=b
%\end{eqnarray}
%
%\item Solved together (still segregated):
%\begin{equation}
%\label{eq:AB:uncoupled}
%\begin{bmatrix}
%\mathbf{A}(y) & 0 \\
%0  & \mathbf{B}(x) \\
%\end{bmatrix}
%\begin{bmatrix}x\\y\end{bmatrix}
%=
%\begin{bmatrix}a\\b\end{bmatrix}
%\end{equation}
%\pause
%\item Coupled solution:
%\begin{equation}
%\label{eq:AB:coupled}
%\begin{bmatrix}
%\mathbf{A}' & \mathbf{A}_y \\
%\mathbf{B}_x  & \mathbf{B}' \\
%\end{bmatrix}
%\begin{bmatrix}x\\y\end{bmatrix}
%=
%\begin{bmatrix}a\\b\end{bmatrix}
%\end{equation}
%\end{itemize}
%\end{frame}
%
%\begin{frame}%[allowframebreaks]
%\setbeamercovered{transparent}
%\frametitle{Coupled approach 2: block coupled approach}
%\begin{itemize}
%\item Alternative formulation:
%\begin{equation}
%\label{eq:C:coupled}
%\mathbf{C}z=c
%\end{equation}
%\pause
%\begin{equation}
%\mathbf{C} = C_{i,j}=\begin{bmatrix}c_{a,a} & c_{a,b} \\ c_{b,a} & c_{b,b}\end{bmatrix}_{i,j}
%\end{equation}
%\begin{equation}
%c =  c_i = \begin{bmatrix}s_a &s_b\end{bmatrix}_i^\top
%\end{equation}
%\begin{equation}
%z=z_i=\begin{bmatrix}x & y \end{bmatrix}_i^\top
%\end{equation}
%\pause
%\item Element in vectors and matrices: vectors and tensors
%\end{itemize}
%\end{frame}

%\begin{frame}%[allowframebreaks]
%\setbeamercovered{transparent}
%\frametitle{Advantages and drawbacks}
%{\bf Coupling advantages:}
%\begin{itemize}
%    \item Possibly faster convergence
%    \item Possibly better algorithms
%\end{itemize}
%{\bf Drawbacks:}
%\begin{itemize}
%    \item Larger amounts of memory 
%    \item New solvers needed
%\end{itemize}
%\end{frame}

\subsection{Application: block coupled incompressible flow solver}
\begin{frame}%[allowframebreaks]
\setbeamercovered{transparent}
\frametitle{Application block coupled incompressible flow solver}
\begin{itemize}\itemsep10pt
    \item Incompressible steady-state Navier-Stokes:
        \begin{equation}
        \nabla\cdot(\Ukrj) = 0 
        \end{equation}
        \begin{equation}
        \label{eq:momentum}
        \nabla\cdot(\Ukrj\Ukrj)-\nabla(\nu\nabla \Ukrj) = -\frac{1}{\rho} \nabla p
        \end{equation}
    \pause
    \item Traditionally segragated solvers (OpenFOAM \OF{simpleFoam})
    \pause
    \item Often slow convergence, pressure equation elliptic behaviour % \footfullcite{jasak:multigrid}
    \pause
    \item Possible remedy: coupled calculation using OpenFOAM 1.6-ext: \OF{BlockLduMatrix}
\end{itemize}
\vfill
\pause
\rule{0.5\textwidth}{0.5pt}\newline
\scriptsize
{\bf Additional sources:}
\begin{itemize}
    %\item \fullcite{jasak:multigrid}
    \item \fullcite{of:block:rusche}
    \item \fullcite{of:block:springer}
    \item \fullcite{of:block:clifford}
\end{itemize}
\end{frame}

\begin{frame}%[allowframebreaks]
\setbeamercovered{transparent}
\frametitle{Theory}
\begin{itemize}
    \item SIMPLE algorithm: pressure equation from the continuity equation, explicit use of velocity (from momentum predictor step)
    \pause
    \item Alternative approach: Rhie-Chow interpolation:
\begin{equation}
\label{eq:cont:rhie}
\displaystyle\sum_\mathrm{faces} \left[\overline{\Ukrj_\mathrm{f}}-\D\left(\nabla \Pp_\mathrm{f}-\overline{\nabla \Pp_\mathrm{f}}\right)\right]\cdot\Sf=0
\end{equation}
\begin{equation}
\label{eq:momentum:semi}
\displaystyle\sum_\mathrm{faces} \left[ \Ukrj \Ukrj- \nu\nabla\Ukrj \right]_\mathrm{f}\cdot\Sf=-\displaystyle\sum_\mathrm{faces} \Pp_\mathrm{f}\Sf
\end{equation}     
\end{itemize}

\vfill
\pause
\rule{0.5\textwidth}{0.5pt}\newline
\scriptsize
{\bf Additional sources:}
\begin{itemize}
    \item \fullcite{cfd:block:darwish}
    \item \fullcite{of:block:mangani}
    \item \fullcite{of:rhie-chow}
\end{itemize}
\end{frame}


\begin{frame}%[allowframebreaks]
\setbeamercovered{transparent}
\frametitle{OpenFOAM implementation I}
\begin{itemize}
    \item Implemented using OpenFOAM-1.6-ext 
    \pause
    \item Block matrix class: \OF{BlockLduMatrix}
    \begin{itemize}
        \item Templated matrix class. Elements of arbitrary size:
        \lstinputlisting[title={\OF{BlockLduMatrix.H}},language=C++,linerange=103-110,firstnumber=103]{/home/klas/OpenFOAM/OpenFOAM-1.6-ext/src/OpenFOAM/matrices/blockLduMatrix/BlockLduMatrix/BlockLduMatrix.H}
        \pause
        \item In this case: manual assembling of matrix coefficients and source for each term in the continuity and momentum equations (Eqs.~\eqref{eq:cont:rhie} and \eqref{eq:momentum:semi})
\pause
\begin{equation*}
\label{eq:cont:rhie}
\displaystyle\sum_\mathrm{faces} \left[\overline{\Ukrj_\mathrm{f}}-\D\left(\nabla \Pp_\mathrm{f}-\overline{\nabla \Pp_\mathrm{f}}\right)\right]\cdot\Sf=0
\end{equation*}
\begin{equation*}
\label{eq:momentum:semi}
\displaystyle\sum_\mathrm{faces} \left[ \Ukrj \Ukrj- \nu\nabla\Ukrj \right]_\mathrm{f}\cdot\Sf=-\displaystyle\sum_\mathrm{faces} \Pp_\mathrm{f}\Sf
\end{equation*}     
    \end{itemize}
\end{itemize}
\end{frame}

\begin{frame}%[allowframebreaks]
\setbeamercovered{transparent}
\frametitle{OpenFOAM implementation II}
\begin{itemize}
    \item For 3D solver: $\mathbf{A}x = f$
    \begin{itemize}
        \item Matrix element:
            \begin{equation}
            a_{i,j} = 
            \begin{bmatrix}
            a_{u,u} & 0 & 0 & a_{p,u} \\
            0 & a_{v,v} & 0 & a_{p,v} \\
            0 & 0 & a_{w,w} & a_{p,w} \\
            a_{u,p} & a_{v,p} & a_{w,p} & a_{p,p} 
            \end{bmatrix}_{i,j}
            \end{equation}
        \item Solution vector:
            \begin{equation}
            x_{i} = 
            \begin{bmatrix}
            u \\ v \\ w \\ p
            \end{bmatrix}_{i}
            \end{equation}
    \end{itemize}
    \pause
    \item Off-diagonal coefficients giving the implicit coupling between pressure and velocity
    \pause
    \item New solvers implemented (\OF{GMRES}, \OF{BiCGStab}, \OF{CG})
\end{itemize}
\pause
\scriptsize
\vfill
\rule{0.5\textwidth}{0.5pt}\newline
{\bf Description of work:}
\begin{itemize}
    \item \fullcite{krj:block:report}
    %\item \fullcite{krj:block:presentation}
\end{itemize}
\end{frame}

\begin{frame}[t]%[allowframebreaks]
\setbeamercovered{transparent}
\frametitle{Case study \OF{pitzDailyFoam}}
\begin{itemize}
    \item New solver benchmarked against \OF{simpleFoam} using the standard \OF{pitzDailyFoam}.
    \pause
    \item Convergence profile (non-turbulent case):
\end{itemize}

\only<2->{
\begin{figure}[h]
\centering
\begin{subfigure}{0.49\textwidth}
    \centering
    \includegraphics<2->[width=1.0\textwidth]{/home/klas/Documents/PhD/Courses/OpenFOAM/Projects/3/Reports/lecture/../final/figures/benchmark/Laminar2.pdf}
    \caption{Iterations}
\end{subfigure}
\begin{subfigure}{0.49\textwidth}
    \centering
    \includegraphics<2->[width=1.0\textwidth]{/home/klas/Documents/PhD/Courses/OpenFOAM/Projects/3/Reports/lecture/../final/figures/benchmark/Laminar2_Time.pdf}
    \caption{Elapsed time}
    \label{fig:benchmark:Laminar:b}
\end{subfigure}
    \caption{Comparison of convergence for \OF{simpleFoam} and \OF{pUCoupledFoam}. Laminar case.}
    \label{fig:benchmark:Laminar}
\end{figure}
}
\end{frame}

\begin{frame}%[allowframebreaks]
\setbeamercovered{transparent}
\begin{itemize}
    \item Convergence profile (turbulent case):
\end{itemize}
\begin{figure}[h]
\centering
\begin{subfigure}{0.49\textwidth}
    \centering
    \includegraphics[width=1.0\textwidth]{/home/klas/Documents/PhD/Courses/OpenFOAM/Projects/3/Reports/lecture/../final/figures/benchmark/Turbulent2.pdf}
    \caption{Iterations}
\end{subfigure}
\begin{subfigure}{0.49\textwidth}
    \centering
    \includegraphics[width=1.0\textwidth]{/home/klas/Documents/PhD/Courses/OpenFOAM/Projects/3/Reports/lecture/../final/figures/benchmark/Turbulent2_Time.pdf}
    \caption{Elapsed time}
\end{subfigure}
    \caption{Comparison of convergence for \OF{simpleFoam} and \OF{pUCoupledFoam}. Turbulent case.}
    \label{fig:benchmark:turbulent}
\end{figure}
\end{frame}


%%%%%%%%%%%%%%%%%%%%%%%%%%%%%%%%%%%%%%%%%%%%%%%%%%%%%%%%%%%%%%%%%%%%%%%%%%%%%%%%
%%%%%%%%%%%%%%%%%%%%%%%%%%%%%%%%%%%%%%%%%%%%%%%%%%%%%%%%%%%%%%%%%%%%%%%%%%%%%%%%
%%%%%%%%%%%%%%%%%%%%%%%%%%%%%%%%%%%%%%%%%%%%%%%%%%%%%%%%%%%%%%%%%%%%%%%%%%%%%%%%

\section{Summary and future outlook}
\begin{frame}%[allowframebreaks]
\setbeamercovered{transparent}
\frametitle{Summary and future outlook}
\textbf{Summary}:
\begin{itemize}
    \item Great possibilities
    \item Successful parallelized multiphysics simulations of nuclear reactor cores
    \item Faster convergence for incompressible flow simulations
\end{itemize}
\pause
\vskip10pt
\textbf{Future developments}:
\begin{itemize}
    \item Multi-grid methods for region and block coupling?
    \item Combining region and block coupling?
    \item .... 
\end{itemize}
\pause
\vskip10pt
\textbf{Community needs}:
\begin{itemize}
    \item Documentation 
    \item Examples (e.g. parallelization of region coupling)
    \item Tutorials
\end{itemize}
\end{frame}

%%%%%%%%%%%%%%%%%%%%%%%%%%%%%%%%%%%%%%%%%%%%%%%%%%%%%%%%%%%%%%%%%%%%%%%%%%%%%%%%
%%%%%%%%%%%%%%%%%%%%%%%%%%%%%%%%%%%%%%%%%%%%%%%%%%%%%%%%%%%%%%%%%%%%%%%%%%%%%%%%
%%%%%%%%%%%%%%%%%%%%%%%%%%%%%%%%%%%%%%%%%%%%%%%%%%%%%%%%%%%%%%%%%%%%%%%%%%%%%%%%


\begin{frame}

{\Huge Thank you! Questions?}
\vfill
{\bf Contact information:}\\
Klas Jareteg\\
klas.jareteg@chalmers.se\\
\url{http://klas.nephy.chalmers.se}
\end{frame}

\end{document}


%%%%%%%%%%%%%%%%%%%%%%%%%%%%%%%%%%%%%%%%%%%%%%%%%%%%%%%%%%%%%%%%%%%%%%%%%%%%%%%
%%%%%%%%%%%%%%%%%%%%%%%%%%%%%%%%%%%%%%%%%%%%%%%%%%%%%%%%%%%%%%%%%%%%%%%%%%%%%%%
%%%%%%%%%%%%%%%%%%%%%%%%%%%%%%%%%%%%%%%%%%%%%%%%%%%%%%%%%%%%%%%%%%%%%%%%%%%%%%%



