\subsection{Poisson-Boltzmann equation}\label{sec:et:pb}
Consider a system consisting of an electrolyte in contact with a
(flat) charged wall.  Under certain assumptions, it is possible to
explicitly determine the charge density in eq. \eqref{eq:et:np} as a
function of the electric potential. E.g. if there is no advection
present and if the system has reached a steady state, i.e. $\partial
\C /\partial t = 0$ and $\ubf = \mathbf{0}$ we have:

\begin{equation}\label{pb_constant_flux}
D\nabla \C + \frac{zq_eD}{k_BT}\C\nabla\psirm = \J_0
\end{equation} 
where $\J_0$ is a constant flux. Due to the steady state
assumption, what the equation above actually says is that the net flux
of charge in the system is constant. Since no charges are wanted to
flow through the wall boundary, the flux is set to zero on the wall
and since the flux is constant it will therefore be zero everywhere in
the liquid, i.e. $\J_0 = 0$.

Considering only a one-dimensional situation with a position variable
$y$ varying in a direction out from the wall into the liquid,
eq. \eqref{pb_constant_flux} reads

\begin{equation}\label{eq:pb_eq_for_C}
\frac{1}{\C} \dfrac{d \C}{d y} + \frac{zq_e}{k_BT} \dfrac{d \psirm}{d
  y} = 0.
\end{equation}

The charge density is determined by solving eq. \eqref{eq:pb_eq_for_C}
for $\C$, i.e. integrating the equation. In order to avoid introducing
additional unknown quantities, the equation is integrated to far away
from the wall where the potential from the EDL is assumed to have
decreased to zero and where the concentrations, $\C^{\infty}$, of the
electrolyte is known.

\begin{equation}
\int_y^{\infty} d\ln( \C(y')) = -\frac{z q_e}{k_BT}\int_y^{\infty}d\psirm(y')
\end{equation}
This gives an expression for $C(y)$:

\begin{equation}\label{eq:C}
\C(y) = \C^{\infty} \exp\left(-\frac{z q_e \psirm(y)}{k_BT}\right).
\end{equation}

In a general case, there may be several species of ions in the
electrolyte, the net charge density, $\rho_e$, is then given by simply
summing up the contributions from the different species:

\begin{equation}\label{eq:rho}
\rho_e = q_e\sum_i z_i \C_i.
\end{equation}

Summarising eqs. \eqref{eq:pb}, \eqref{eq:C} and \eqref{eq:rho} gives
the Poisson-Boltzmann equation in one dimension

\begin{equation}\label{eq:pb_real}
\dfrac{d^2\psirm(y)}{dy^2} = -\frac{q_e}{\epsilon_r \epsilon_0}\sum_i z_i
\C_i^{\infty} \exp\left(-\frac{z_i q_e \psirm(y)}{k_BT}\right).
\end{equation}

\subsubsection{The Debye–Hückel approximation}
Historically, the non-linear nature of eq. \eqref{eq:pb_real}
complicated for those wanting to solve it. This was a major difficulty
in the past when the computational power at hands were rather
limited. A linearisation is therefore sometimes done, this linear
version of the PB equation is often referred to as the Debye–Hückel
approximation. The solution of the linearisation gives, something to
compare with and is usually used when defining a characteristic length
scale of the EDL.

For a 1:1 electrolyte solution with an equal amount of positive and
negatively charged ions, eq. \eqref{eq:pb_real} reduces to

\begin{equation}
\frac{d^2\psi(x)}{dx^2} = \frac{2\C^{\infty}q_ez}{\epsilon_r
  \epsilon_0}
\sinh\left(\frac{z q_e \psi(x)}{k_BT}\right).
\end{equation}
and the linearised equation is

\begin{equation}
\frac{d^2\psi(x)}{dx^2} = \frac{2\C^{\infty}q_e^2z^2}{\epsilon_r
  \epsilon_0 k_B T} \psi(x) = \kappa^2 \psi(x)
\end{equation}
where $\kappa^{-1}$ is the Debye length which is where the exponential
solution has decayed to $e^{-1}$ of the boundary value. This quantity
gives therefore a measure for the characteristic thickness of the EDL.

\nomenclature{PB, P-B}{Poisson-Boltzmann model/equation}
%% In this work, flows of ionic solution will be studied and the
%% assumption with thermodynamical equilibrium does not apply. However
%% for low-speed flows the model may still be a decent approximation,
%% which will be investigated.

%% The second assumption, may also stay unfulfilled in some cases
%% investigated here. The fluid in contact with the wall must be of
%% substantial size in relation to the EDL thickness. There will be
%% cases where the choice of $\zeta$ potential in combination with thin
%% channels will make this assumption not fulfilled.

%% Since the PB equation is unable to model the system of
%% interest, a different approach will be presented. However, throughout
%% this work, references and comparisons with the PB model w
%% ill be made. 


%intro

%liten härledning av högerledet 
%diffusion by conc grad. = grad of
%potential <==> termodynamic equilibrium
%chem pot definerad som....
% eq.
% boundary conditions
%antagaganden !!!


%% A simple and commonly used approach for determining potentials (and
%% charge distributions) in systems with present EDLs is by solving
%% eq. \eqref{eq:pb} with a charge distribution of Boltzmann
%% type. Here follows a brief derivation of this term together with some
%% discussion on the assumptions made.

%% The fundamental assumption that the derivation of the charge
%% distribution is based on, is the fact that the system is assumed to be
%% under thermodynamical equilibrium. I.e. forces, acting on the ions,
%% due to chemical diffusion from concentration gradients and from the
%% electrical field are therefore balancing each other. In one dimension:

%% \begin{equation}\label{eq:dif_elec_forces}
%% \frac{d \mu_i}{dx} = -z_i q_e\frac{d\psi}{dx}
%% \end{equation}
%% where $\mu_i$ is the chemical potential for species $i$, $z_i$ is the
%% relative charge of species $i$, $q_e$ the fundamental charge and
%% $\psi$ is the EDL potential. The chemical potential is given by
%% \cite{ren}:

%% \begin{equation}
%% \mu_i = \mu_i^{\infty} + k_BT\ln n_i
%% \end{equation}
%% where $\mu_i^{\infty}$ is a reference value for the chemical potential,
%% here the potential value far from the charged wall is used, $k_BT$ is
%% the thermal energy and $n_i$ is the ion concentration of species
%% $i$. This expression plugged into eq. \eqref{eq:dif_elec_forces} gives

%% \begin{equation}\label{eq:eq_for_ni}
%% \frac{d \ln(n_i)}{dx} = - \frac{z_i q_e}{k_BT}\frac{d \psi}{dx}.
%% \end{equation}
