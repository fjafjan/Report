\label{sec:jeffery}
The Jeffery orbits describe the orientational motion of an ellipsoidal particle in Stokes flow. The general equations of motion for any ellipsoid in shear flow were found by Jeffery\cite{Jeffery} who also solved these equations of motion for axisymmetric ($a_1 \ne a_2 = a_3$) ellipsoidal particles. Solutions for asymmetric particles were found numerically by Yarin \emph{et al.}~\cite{Yarin} who also rewrote the equations in a different but equivalent form. The equations of rotational motion for a triaxial particle in shear flow are in the form of Yarin \emph{et al.}

\begin{subequations}\label{eq:jeffrey}
\begin{align}
\frac{d\theta}{dt} 	&= (g_2 \sin \psi + g_3 \cos \psi ) \sin \theta, \\
\frac{d\phi}{dt} 	&= \tfrac{1}{2} + g_3\sin \psi - g_2 \cos \psi,\\
\frac{d\psi}{dt}	&= g_1 + (g_2\cos \psi - g_3\sin \psi) \cos \theta. \\
\end{align}
\end{subequations}

\noindent The functions  $g_i$ are defined as

\begin{subequations}
\begin{align}
g_1 &= \frac{a_2^2 - a_3^2}{2(a_2^2 + a_3^2)} 
		\left(-\tfrac{1}{2}(\cos^2 \theta + 1 )\sin 2\phi \sin 2\psi + \cos\theta \cos 2\phi \cos 2\psi \right), \\
g_2 &= \frac{a_3^2 - a_1^2}{2(a_1^2 + a_3^2)}
		\left( -\cos\theta \sin 2\phi \sin\psi  +  \cos 2\phi \cos\psi \right), \\
g_3 &= \frac{a_1^2 - a_2^2}{2(a_1^2 + a_2^2)}
		\left( \cos\theta \sin 2\phi \cos\psi + \cos 2\phi \sin\psi \right).
\end{align}
\end{subequations}

\noindent Here $(\phi, \theta, \psi)$ are the Euler angles seen in Figure \ref{fig:eulerangles}. 
%Note that Yarin uses $a_x, a_y, a_z$ in place of $a_1, a_2, a_3$.

Note that the eq. (\ref{eq:jeffrey}) uses the coordinate system from Yarin \emph{et al.}\cite{Yarin} which differ from the one used in this thesis, details are discussed in Johansson \cite{AntonThesis}. Numerical solutions for three different initial conditions for an asymmetric particle are shown in Figure \ref{fig:orbitparams}.

\begin{figure}[H]
\centering
\begin{subfigure}[b]{0.45\textwidth}
\includegraphics[width=\textwidth]{figures/theory/map.pdf}
\caption{A poincare map}\label{fig:orbitmap}
\end{subfigure}\hspace{1em}%
\begin{subfigure}[b]{0.5\textwidth}
\includegraphics[width=\textwidth]{figures/theory/orbit.pdf}
\caption{The time series for the components \\ of the unit vector.}\label{fig:orbitparams}
\end{subfigure}
\caption{A Poincare map and three different orbits for a simulated particle with $\lambda=7$ and $\epsilon=0.05$. The three orbits highlight the three different kinds of motion, the quasi-periodic sign changing orbit in blue, the quasi-periodic sign preserving orbit in red and the periodic orbit in green. We see that while $n_x$ and $n_y$ look qualitatively similar but differ in amplitude for the different orbits, $n_z$ shows three different types of behaviour}
\label{fig:orbittypes}
\end{figure}



Looking at Figure \ref{fig:orbitparams} we see that $n_x$ and $n_y$ are periodic, corresponding to periodic rotation around the $z$-axis. We refer to these rotations as flips. The period $\eta$ of $n_x$ and $n_y$ is for an axisymmetric particle \cite{Jeffery}

\begin{equation}\label{eq:flipRate}
\eta = 2\pi \left( \lambda + \frac{1}{\lambda} \right)\frac{1}{\kappa},
\end{equation}

\noindent where $\kappa$ is the shear rate. 

The time evolution of $\theta$ and $\psi$ for different initial conditions can be plotted in a Poincaré map, also known as a Surface-of-Section (S.O.S.)~\cite{poincare}. This plots the $\psi$ and $\theta$ coordinates each time $\phi = 0$. The successive points on the Poincaré map for each initial condition move and explore certain regions of the surface of section. This region is referred to as the \emph{orbit}. 

For a particle with an asymmetry $\epsilon$ in the range $\left[0.01-0.05\right]$ there are three classes of orbits, depending on the initial condition $\theta_0$.

\begin{enumerate}
\item \textbf{Periodic}: $\left|\theta_0\right| \approx 1$ in which there is little variation and the particle is largely periodic with fluctuations too small to measure.
\item \textbf{Quasi-periodic sign preserving}: For $\left|\theta_0\right|> \theta_b$ the amplitude of $\cos(\theta)$ changes noticeably but does not change sign. Here $\theta_b$ is a breaking point that changes for different $\epsilon$.
\item \textbf{Quasi-periodic sign changing}: For small $\left|\theta_0\right| < \theta_b$ the amplitude of $\cos(\theta)$ changes noticeably and changes sign. 
\end{enumerate}

Simulations of these three different types of orbits for a particle with $\lambda=7$ and $\epsilon=0.05$ are illustrated in Figure \ref{fig:orbittypes}. Figure \ref{fig:orbittypes} shows the orbits on the S.O.S. and Figure \ref{fig:orbitparams} shows the components of $\mathbf{n}$ as a function of time. We can see that while $n_x$ and $n_y$ are periodic, albeit with different amplitudes, the  behaviour of $n_z$ is significantly different for the different orbits. The $n_z \approx 1$  orbit shown in green is constant on the S.O.S and is simply periodic over time. The sign preserving quasi-periodic orbit in red is bent on the S.O.S. We can see in the time series that it is doubly periodic as it peaks with a fixed period but the amplitude of the peaks vary periodically themselves. The sign changing quasi-periodic orbit in blue also peaks periodically with varying peak amplitude but these also change sign, again with a fixed period.


For larger asymmetries $\epsilon > 0.05$ there are chaotic orbits that explore a larger region of the S.O.S. The simulations used do not have long enough time evolutions for the chaotic orbits to fill the regions in the way that quasi-periodic or periodic orbits fill the one dimensional regions that define their orbits. This results in the chaotic orbits appearing as a 'sea' of dots instead of filled lines. Chaotic orbits can been seen around the quasi-periodic sign changing orbits in Figure \ref{fig:orbitmap4}.

\begin{figure}[H]
\centering
\begin{subfigure}[3a]{0.40\textwidth}
\includegraphics[width=\textwidth]{figures/theory/7-1-1.pdf}
\caption{Poincare map for $\lambda = 7, \epsilon = 0$.}\label{fig:orbitmap1}
\end{subfigure}\hspace{1em}%
\begin{subfigure}[3b]{0.40\textwidth}
\includegraphics[width=\textwidth]{figures/theory/7-1o01-1.pdf}
\caption{Poincare map for $\lambda = 7, \epsilon = 0.01$.}\label{fig:orbitmap2}
\end{subfigure} \\
\begin{subfigure}[3a]{0.40\textwidth}
\includegraphics[width=\textwidth]{figures/theory/7-1o05-1.pdf}
\caption{Poincare map for $\lambda = 7, \epsilon = 0.05$.}\label{fig:orbitmap3}
\end{subfigure}\hspace{1em}%
	\begin{subfigure}[3b]{0.40\textwidth}
\includegraphics[width=\textwidth]{figures/theory/7-1o25-1.pdf}
\caption{Poincare map for $\lambda = 7, \epsilon = 0.25$.}\label{fig:orbitmap4}
\end{subfigure} 
\caption{Four Poincare maps for different $\epsilon$. Already at $\epsilon = 0.01$ there are noticeably quasi-periodic 
orbits around the centre at $\cos(\theta) \approx \psi \approx 0$ but it is also a significantly larger region for $\epsilon = 0.05$. For $\epsilon = 0.25$ we can see chaotic orbits surrounding the circular orbits in the centre that appear as a 'sea' of dots.} %q Note that some wavelike pattern can appear to exist in the figure \ref{fig:orbitmap2} and  \ref{fig:orbitmap2}, this is caused by aliasing/compression issues with printing several curved lines close together.}\label{fig:orbitmaps}
\end{figure}

\subsection{Winding number} \label{sec:winding}
The quasi-periodic orbits are also referred to as doubly-periodic~\cite{Yarin}. It is called doubly-periodic to emphasize the fact that the amplitude of the short period $\theta_2$ also varies periodically with period $\theta_1$. The ratio between the two periods is referred to as the winding number $\omega$ referring to the winding around a unit torus. The shorter period $\theta_2$ corresponds to rotations around the small cross section and the longer period $\theta_1$ corresponds to rotations around the large circumference of the torus. The winding number is defined as \cite{introchaos}

\begin{equation}
\omega_{def}  = \lim\limits_{n \rightarrow \infty} \frac{f^n(\xi) - \xi_0}{n}.
\end{equation}

where $\xi$ is the angle around the large circumference of the torus and $f^n(\xi)$ is the shift caused in $n$ rotations around the smaller axis. 
Applying this to our double-periodic orbits $\xi$ would be the angle with period $\theta_1$ and $n$ the number of flips with period $\theta_2$. 
Since we cannot measure $\xi$ I consider it more comprehensible to consider the inverse winding number: The number of flips necessary to 
complete one period of $\xi$. This can be measured as is shown in Figure \ref{fig:windingDef}. If we measure over several $\theta_1$ peaks the average winding number approximates the real one. We refer to the this inverse winding number as $\omega$ and it is given by

\begin{equation}\label{eq:winding}
\omega = \frac{\theta_1}{\theta_2}.
\end{equation}

\noindent The winding number of the quasi-periodic sign preserving orbit from Figure \ref{fig:orbittypes} and the way we approximate $\theta_1$ (and thereby $\omega$) is illustrated in Figure \ref{fig:windingDef}.

\begin{figure}[H]
\begin{center}
\includegraphics[width=0.7\textwidth]{figures/theory/WindingNrFixed2.pdf}
\end{center}
\caption{The quasi-periodic sign preserving orbit from Figure \ref{fig:orbitparams} over a longer time, highlighting the short period $\theta_2$ which is simply the period of $\phi$ and the longer period $\theta_1$. We find the inverse winding number as the ratio between the longer and shorter periods.}
\label{fig:windingDef}
\end{figure}

%This can also be thought of as the number of intersections on the surface of section before coming back to the initial condition, divided by the number of laps. A lap for a circular orbit is a rotation around the center whereas for a flat orbit it is moving along length of the orbit. asdasd, see figure MAKE A FIGURE. 
The winding number is the same for any point along a quasi-periodic orbit on a Poincaré map but it is different for different orbits as well as for different asymmetries. The winding numbers for orbits along $\psi=0$ for $\epsilon=\{0.01, 0.05, 0.10\}$ can be seen in Figure \ref{fig:windingdifferent}. The difference in winding number for the sign changing orbits almost a factor 2. This means that if we can measure the winding number it allows us to approximate the asymmetry of the particle. Without looking at the winding number we cannot determine if a particle with an orbit close to $n_z = 0$ and small variations in amplitude is close to symmetric or not.
 
\begin{figure}[H]
\begin{center}
\includegraphics[width=0.7\textwidth]{figures/theory/WindingTrend.png}
\end{center}
\caption{The winding number for $\psi_0 = 0$ and $\lambda = 7$ as a function of $\cos(\theta_0)$ for three different asymmetries, $\epsilon = 0.01$, $\epsilon = 0.05$ and $\epsilon = 0.10$. The sharp edge that occurs centred around zero is where the sign changing orbits end and sign preserving orbits begin, denoted by $\theta_b$. We see that a lower asymmetry leads to a sharper difference between the sign changing and the sign preserving orbits and that higher asymmetry in general leads to lower winding numbers on average.}
\label{fig:windingdifferent}
\end{figure}
