As mentioned above this thesis is a continuation of previous work \cite{AntonThesis, JonasExperiment, Mishra}. There were a number of problems that needed to be solved in order to improve the results. They can be summarized as:
\begin{enumerate} \label{list:problems}
	\item Particles 
	\begin{itemize}
		\item Very few particles used in previous experiments were sufficiently symmetric to have periodic orbits or even quasi-periodic orbits. Most were visibly bent or uneven, see figure \ref{fig:oldparticles}
		\item The average aspect ratio of the particles was very high which meant there were very few flips along a stretch.
		\item The width of the particles was highly irregular and the resolution of the microscope did not allow for accurate measurements of it. This meant it was difficult to accurately estimate the aspect ratio.
		\item The particles could not be trapped with an optical tweezers due to low transmittance.
	\end{itemize}
	\item The PDMS in the channel was jagged which gave rise to noise unless the focus was in a very narrow band.
	\item Manual tracking of particles was time consuming.
	\item Bubbles are difficult to avoid when setting up the experiment.
\end{enumerate}

