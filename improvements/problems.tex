As previously mentioned this thesis is a continuation of work done by Mehlig, Einarsson and Mishra et al \cite{AntonThesis, JonasExperiment, Mishra}. Their results were promising but there were a number of key limitations and problem that we want to solve to improve the results and the ease of getting results. Roughly they are
\begin{enumerate}
	\item The particles
	\begin{itemize}
		\item Very few particles are symmetric, most are visibly bent or uneven, see figure \ref{fig:oldparticles}
		\item Few particles have a low enough aspect to be useful.
		\item Have greatly varying and difficult to measure width meaning the aspect ratio cannot be estimated well
		\item Cannot be trapped with an optical tweezer due to low transmittance
	\end{itemize}
	\item The PDMS in the channel is very jagged which causes a great deal 
			of noise unless the focus is in a very narrow band
	\item Manual tracking of particles is time consuming and mentally draining.
	\item Bubbles are difficult to avoid when setting up the experiment
\end{enumerate}

