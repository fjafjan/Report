As previously mentioned this thesis is a continuation of work done by Mehlig \emph{et al}, Einarsson \emph{et al} and Mishra \emph{et al} \cite{AntonThesis, JonasExperiment, Mishra}. Their results were promising but there were a number of key limitations and problems that need to be solved in order to improve the results. They can be summarized as:
\begin{enumerate} \label{list:problems}
	\item The particles 
	\begin{itemize}
		\item Very few particles used in previous experiments were sufficiently symmetric to have quasi periodic orbits. Most were visibly bent or uneven, see figure \ref{fig:oldparticles}
		\item The average aspect ratio of the particles was very high which meant there were very few flips along a stretch.
		\item The width of the particles could not be measured, is not uniform and very small which makes estimates of the aspect ratio hard.
		\item The particles could not be trapped with an optical tweezers due to low transmittance.
	\end{itemize}
	\item The PDMS in the channel was very jagged which caused a great deal 
			of noise unless the focus was in a very narrow band.
	\item Manual tracking of particles was time consuming and mentally draining.
	\item Bubbles are difficult to avoid when setting up the experiment
\end{enumerate}

