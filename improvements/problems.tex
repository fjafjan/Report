As previously mentioned this thesis is a continuation of work done by Mehlig, Einarsson and Mishra et al \cite{AntonThesis, JonasExperiment, Mishra}. Their results were promising but there were a number of key limitations and problem that we want to solve to improve the results and the ease of getting results. They can be summarized as:
\begin{enumerate}
	\item The particles
	\begin{itemize}
		\item Very few particles are symmetric, most are visibly bent or uneven, see figure \ref{fig:oldparticles}
		\item The average aspect ratio is very high which means there will be very few flips along a stretch.
		\item The width cannot be measured and is not uniform which makes estimates of the aspect ratio hard.
		\item Cannot be trapped with an optical tweezer due to low transmittance.
	\end{itemize}
	\item The PDMS in the channel is very jagged which causes a great deal 
			of noise unless the focus is in a very narrow band
	\item Manual tracking of particles is time consuming and mentally draining.
	\item Bubbles are difficult to avoid when setting up the experiment
\end{enumerate}

