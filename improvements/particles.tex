\label{sec:particle_improves}
\begin{figure}[H]
\centering
\begin{subfigure}[b]{0.45\textwidth}
\includegraphics[width=0.9\textwidth]{figures/improvements/oldparticle2.png}
\caption{Particle 13 from July 2012}
\end{subfigure}
\begin{subfigure}[b]{0.45\textwidth}
\includegraphics[width=0.9\textwidth]{figures/improvements/oldparticle3.png}
\caption{Particle 22 from July 2012}
\end{subfigure}
\caption{Two fairly typical particles from the previous setup. Note that these are still selected from the total pool of particles for being relatively symmetric and yet are noticeably bent.}
\label{fig:oldparticles}
\end{figure}



The polymer particles were replaced with glass particles from Nippon Glass, Japan \cite{Particles}. The new particles are made 
from LCD spacing rods that are broken into pieces. This means that they are essentially broken cylinders with very 
homogeneous widths but quite disparate lengths. Two different batches of particles have been used, one with a $3\mu m$ diameter and one batch with $5 \mu m$ diameter. All the measurements presented in the results section are from the $3 \mu m$ width particles. 

The symmetries of the particles were investigated with the help from Stefan Gustafsson by taking images with an 
ESEM (Environmental Scanning Electron Microscope) shown in figure \ref{fig:particlepictures}. We see that the 
particles are uniformly smooth along the sides but have varyingly jagged edges causing different degrees of asymmetry. 

In particular figure \ref{fig:roundparticle} shows a top down view of a particle clearly showing a very circular shape 
with no discernible asymmetry whereas figure \ref{fig:particlepictures} show the jagged edge of several particles. 


\begin{figure}[H]
\centering
\begin{subfigure}[3a]{0.40\textwidth}
\includegraphics[width=\textwidth]{figures/method/semizoomed.png}
\caption{A detailed view \\ of a number of particles.}
\end{subfigure}\hspace{1em}%
\begin{subfigure}[3b]{0.40\textwidth}
\includegraphics[width=\textwidth]{figures/method/zoomedbroken.png}
\caption{The jagged edge of a particle \\ in detail.}
\end{subfigure}
\caption{Pictures of the glass particles that were used. Their width is highly uniform and there is a noticeable variance is asymmetry. Some particles show very clearly jagged edges while other appear very smooth. This suggests that they should have quite different $\epsilon$ and then exhibit quite different behaviour.}
\label{fig:particlepictures}
\end{figure}
 
\begin{figure}[H]
\centering
\begin{subfigure}[3a]{0.40\textwidth}
\includegraphics[width=\textwidth]{figures/method/symmetric.png}
\caption{What appears to be a highly \\ symmetric particle.}\label{fig:symmetricparticle}
\end{subfigure}\hspace{1em}%
\begin{subfigure}[3b]{0.40\textwidth}
\includegraphics[width=\textwidth]{figures/method/round.png}
\caption{A top down view of a particle.}\label{fig:roundparticle}
\end{subfigure}
\caption{Pictures highlighting the roundness of the particles as well as the apparent symmetry of some particles. It should be noted that although there are no apparent rough edges there was no way to rotate a sample so there might very well be asymmetries on the side of the particle that we cannot see.}
\label{fig:particlepictures2}
\end{figure}

Figure \ref{fig:roundparticle} and \ref{fig:symmetricparticle} are the same as can be seen in \cite{alexanderThesis} figure 5.2(c) and 5.2(b) respectively. 

While these particles seemingly satisfy the symmetry conditions they are made of glass with a density of approximately 
\unit[2.57]{g/cm$^3$} at \unit[20]{C$^\circ$}. This is significantly higher than that of water with a density of 
\unit[1]{g/cm$^3$} at \unit[20]{C$^\circ$} and glycerol with a density of \unit[1.5]{g/cm$^3$}. Thus to correct for the 
density and limit sinking or floating the water soluble Sodium metatungstate which at \unit[20]{C$^\circ$} has maximum 
density of \unit[2.94]{g/cm$^3$} is added to the liquid. To increase the viscosity of the liquid around 8\% glycerol is added and the liquid 
was measured using a MCR 302 rheometer to have a dynamic viscosity of \unit[$24\cdot 10^{-3}$]{Pa s}.
