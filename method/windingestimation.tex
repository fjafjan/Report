	\label{sec:windingEstimation}
As discussed in section \ref{sec:winding}, estimating the winding number for different types of orientational orbit for one particle allows for a better estimation of $\epsilon$. In order to estimate the winding number for a measured 
particle we must identify the two periods $\theta_1$ and $\theta_2$ from Figure \ref{fig:windingDef}. 


\begin{figure}
\centering
\includegraphics[width=0.7\textwidth]{figures/method/nzNx0.pdf}
\caption{The stars are plotted at the same distances in the $n_x$ and $n_z$ plots. We see that zeros of $n_x$ and maxima of $n_z$ occur almost exactly at the same points. These points are referred to as $\mathbf{P}_z$}
\label{fig:nzNx0}
\end{figure}

The maxima with the shorter period $\theta_2$ are located where $n_x = 0$ as is seen in Figure \ref{fig:nzNx0}. We denote the set of these points $P_z$. The longer period $\theta_1$ is the periodicity of $P_z$, marked as red points in Figure\ref{fig:nzNx0}. To estimate the winding number we want to locate the maxima $M$ and minima $m$ in $P_z$ that occur with period $\theta_1$ in the same way we do to find $\theta_2$ in $n_z$. Unfortunately the height of peaks is noisy and there are few data points for each measurement as the channel is of finite length only allowing a few dozen flips. This means averaging cannot be used to reduce the noise.This means we have no algorithmic means to find $\theta_1$. 

Instead we select a number of maxima $M_1, M_2 ... M_p$ and minima $m_1, m_2, ..., m_q$ from $\mathbf{P}_z$ . We denote their index in $n_z$ as $I^M_1, I^M_2, 
..., I^M_p$ for the maxima and $I^m_1, I^m_2, ..., I^m_q$ for the minima. We then find $\theta_1$ as the mean distance between successive maxima $\overline{d_M}$ and successive minima $\overline{d_m}$, 

\begin{align}
\overline{d_M} &= \frac{1}{p-1} \sum\limits_{j=1}^{p} I^M_{j+1} - I^M_{j} \\
\overline{d_m} &= \frac{1}{q-1} \sum\limits_{j=1}^{q} I^m_{j+1}- I^m_{j}\\
\hat{\theta_1}   &= \frac{\overline{d_M} + \overline{d_m}}{2}.
\label{eq:winding2}
\end{align}

In the case that we only have 1 maxima and minima eq. \ref{eq:winding2} can't be calculated so we assume that the distance between a maxima and minima is half a period, i.e.

\begin{equation}
\hat{\theta_1} = 2\left| I^M_1 - I^m_1 \right|
\end{equation}