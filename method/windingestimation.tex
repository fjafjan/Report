\label{sec:windingEstimation}
As discussed in section \ref{sec:winding}, estimating the winding number for different types of orbit for one particle 
should allow for a rather accurate estimation of $\epsilon$. So in order estimate the winding number for a measured 
particle we must identify where the $\theta_1$ maxima and minima from figure \ref{fig:windingDef} occur. First, the 
$\theta_2$ maxima is located by where $n_x = 0$. A plot of such 
points can be seen in figure \ref{fig:nzNx0} where we match each $n_z$ peak with an index $i$. Unfortunately data is 
too noisy to allow for simple algorithmic approaches 
to finding good local maxima. Instead we select a number of maxima $M_1, M_2 ... M_p$ with peak index $I^M_1, I^M_2, 
..., I^M_p$  and minima $m_1, m_2, ..., m_q$ with peak index $I^m_1, I^m_2, ..., I^m_q$.

\begin{figure}
\centering
\includegraphics[width=0.7\textwidth]{figures/method/nzNx0.pdf}
\caption{The zeros of nx and maxes of nz}
\label{fig:nzNx0}
\end{figure}

We can estimate the winding number $\hat{w}$ as the average distance between the peak indices for the maxima and minima 
i.e. 

\begin{align}
\overline{d_M} &= \frac{1}{p-1} \sum\limits_{j=1}^{p} I^M_{j+1} - I^M_{j} \\
\overline{d_m} &= \frac{1}{q-1} \sum\limits_{j=1}^{q} I^m_{j+1}- I^m_{j}\\
\hat{w}   &= \frac{\overline{d_M} + \overline{d_m}}{2}.
\label{eq:winding2}
\end{align}


In the case that we only have 1 maxima and minima we use instead that the distance between maxima and minima should be 
half a period, i.e.

\begin{equation}
\hat{w} = \left| I^M_1 - I^m_1 \right| \cdot 2
\end{equation}