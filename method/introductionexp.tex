To measure Jeffery orbits we need to have an experimental setup with which to measure. The one used in this thesis is an iteration of the one used by Einarsson \emph{et al.}~\cite{JonasExperiment}, Johansson \cite{AntonThesis}, and Mishra \emph{et al.}~\cite{Mishra}. In this chapter we describe the setup and why it is designed the way it is. We also discuss the improvements over previous iterations, as well as new problems that have arisen from the changes made. In particular we describe a tracking algorithm that was implemented to make gathering data easier.