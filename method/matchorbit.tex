\label{sec:matchorbit}
To find the best matching orbit from a poincare map we again utilize the $n_z$ peaks where $\phi = 0$ which we call $\mathbf{P_z}$ from figure \ref{fig:nzNx0}. $\mathbf{P_z}$ still is $N$ long but now we are concerned with simply matching them with the peaks from theoretical orbits as opposed to estimating a winding number.

%To find the best matching theoretical orbit we calculate the least square distance of $\mathbf{P_z}$ against the theoretical peaks for $\epsilon \in [0.01, 0.02, ..., 0.1]$ and for $\cos(\theta) \in [-1,1]$ 
%and for $\psi \in [-\frac{pi}{2},\frac{pi}{2}]$. 

To find the best matching theoretical orbit we computed the least square distance for all orbits on all face maps with $\epsilon \in [0.01, 0.02, ..., 0.1]$. If we for each orbit denote our theoretical series of $n_z$ peaks $\mathbf{Q_z}(\theta, \epsilon)$ of length $M >> N$ we can assign a score function $S(\theta, \epsilon)$ as

\begin{equation}
S(\mathbf{P}, \theta, \epsilon) = \left| \mathbf{P_z} - \mathbf{Q_z(\theta, \epsilon)} \right|^2
\end{equation}

However matching against different $\theta$ and $\epsilon$ is not enough to get small mean square distance as the phase 
of the time series might be different. It can be assumed that $P_z$ will be at least half a period of the quasi period 
$\theta_2$ since otherwise we would have no reason to think there is quasi periodic behaviour and trying to match an 
orbit would be fairly pointless. If $Q_z^{(i)}$ denotes the series of peaks starting at i and  $P_z$ is at 
least half a period we can calculate the score for starting at a certain starting index $S(\theta, \epsilon, i)$

\begin{equation}
S(\mathbf{P}, \theta, \epsilon, i) = \left| \mathbf{P_z} - \mathbf{Q_z^{(i)}} \right|^2.
\end{equation}

$\epsilon$ will not actually change for a single particle over $r$ different measurements $P_1, P_2, ..., P_r$, whereas $\theta$ and $i$ (the phase) will. We find the best $\theta$ and $i$ as
\begin{equation}
\hat{S}(\mathbf{P}_j, \epsilon) =  \min(S(\mathbf{P}_j, \theta, \epsilon, i))
\end{equation}
and find the best $\epsilon$ using

\begin{eqnarray}
\epsilon_{best} = \min(\sum\limits_{j=1}^{r} \hat{S}(\mathbf{P}_j, \epsilon)^2).
\end{eqnarray}

% Refer to some matched plots once I have added those. I THINK basically everythig should already be in the saved plots and data folder. 