\label{sec:matchorbit}
To verify that the theoretical orbits from Section \ref{sec:jeffery} have been measured we want to match the measurements to theoretical orbits.

To find the best matching orbit from a Poincaré map for a measurement we again utilize $\mathbf{P}_z$, the points where $n_z$ peaks and $n_x=0$. We denote the length of $\mathbf{P}_z$ as $N$. We are concerned with matching them with the peaks from theoretical orbits.

%To find the best matching theoretical orbit we calculate the least square distance of $\mathbf{P_z}$ against the theoretical peaks for $\epsilon \in [0.01, 0.02, ..., 0.1]$ and for $\cos(\theta) \in [-1,1]$ 
%and for $\psi \in [-\frac{pi}{2},\frac{pi}{2}]$. 

To find the best matching theoretical orbit for a measurement we compute the least square distance between $\mathbf{P}_z$ and all orbits on all phase maps 
with $\epsilon$ in the range $\left[0.01, 0.02, ..., 0.1\right]$ for 200 consecutive initial $\psi$ (the $x$-axis on the Poincaré map). If we for each orbit denote 
the theoretical series of $n_z$ peaks as $\mathbf{Q}_z(\theta, \epsilon)$. We make sure that the length of $\mathbf{Q}_z(\theta, \epsilon)$ be twice that 
of $\mathbf{P}_z$ which guarantees that we always find the correct phase. We define $\mathbf{Q}_z(\theta, \epsilon, i)$ to be the $n_z$ series 
$\mathbf{Q}_z(\theta, \epsilon)$ starting at index $i$. We assign a score function $S(\theta, \epsilon, i)$ as

\begin{equation}
S(\mathbf{P}_z, \theta, \epsilon, i) = \frac{1}{N}\left| \mathbf{P}_z - \mathbf{Q}_z^{(i)}(\theta, \epsilon) \right|^2.
\end{equation}

\noindent An example experimental $P_z$ series matched to theoretical data is seen in Figure \ref{fig:particleB2match}.

\begin{figure}[H]
\centering
\includegraphics[width=0.7\textwidth]{figures/results/particleB/October_1_Particle_4_run_4match.pdf}
\caption{The upper figure shows the experimental $n_z$ peaks $\mathbf{P}_z$ and the theoretical peaks $\mathbf{Q}_z$ for the best matching orbit. The lower plot shows where what section of the theoretical time series was used for matching, ie what $i$ from section \ref{sec:matchorbit} was chosen.}
\label{fig:particleB2match}
\end{figure}



\noindent Over $r$ different measurements $\mathbf{P}_z^{(1)}, \mathbf{P}_z^{(2)}, ..., \mathbf{P}_z^{(r)}$ we expect  the initial conditions $\theta$ and $i$ (the phase) to change. The asymmetry $\epsilon$ does not change between measurements. So for each $\epsilon$ we find the best $\theta$ and $i$ using the temporary score function $\hat{S}(\mathbf{P}^{(j)}_z, \epsilon)$ 
\begin{equation}
\hat{S}(\mathbf{P}^{(j)}_z, \epsilon) =  \min(S(\mathbf{P}^{(j)}_z, \theta, \epsilon, i)) 
\end{equation}

\noindent and  then find the best $\epsilon$ for a particle using

\begin{eqnarray}
\epsilon_{best} = \min \left(\sum\limits_{j=1}^{r} \frac{\hat{S}(\mathbf{P}^{(j)}_z, \epsilon)^2)}{N^{(j)}} \right).	
\end{eqnarray}


It is important to note that this matching has a theoretical limitation. The asymmetry of our cylindrical particles is different from the asymmetry in the triaxial particles used in the theoretical models. The triaxial particles still have a rotational symmetry for rotations of $\pi$ around any of the major/minor axes. This is not the case with the asymmetric cylindrical particles, which have no rotational symmetry. We therefore have to make the assumption that the difference between these two types of asymmetry is not significant and further theoretical work might reject this assumption.

% Refer to some matched plots once I have added those. I THINK basically everythig should already be in the saved plots and data folder. 