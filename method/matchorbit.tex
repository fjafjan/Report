\label{sec:matchorbit}
To verify that the theoretical Jeffery orbits from \ref{sec:jeffery} are present we want to match the measurements to theoretical orbits.

To find the best matching orbit from a Poincaré map for a measurement we again utilize $P_z$, the points where $n_z$ peaks and $n_x=0$. We denote the length o f$\mathbf{P_z}$ as $N$. We are concerned with matching them with the peaks from theoretical orbits.

%To find the best matching theoretical orbit we calculate the least square distance of $\mathbf{P_z}$ against the theoretical peaks for $\epsilon \in [0.01, 0.02, ..., 0.1]$ and for $\cos(\theta) \in [-1,1]$ 
%and for $\psi \in [-\frac{pi}{2},\frac{pi}{2}]$. 

To find the best matching theoretical orbit for a measurement we compute the least square between $P_z$ and all orbits on all phase maps with $\epsilon \in [0.01, 0.02, ..., 0.1]$ for 200 consecutive different initial $\psi$ (the $x$-axis on the Poincare map). If we for each orbit denote the theoretical series of $n_z$ peaks as $\mathbf{Q_z}(\theta, \epsilon)$. We let $\mathbf{Q_z}(\theta, \epsilon)$ be of length $M > 2N$ and define $\mathbf{Q_z}(\theta, \epsilon, i)$ to be the $n_z$ series $\mathbf{Q_z}(\theta, \epsilon)$ starting at index $i$. We assign a score function $S(\theta, \epsilon, i)$ as

\begin{equation}
S(\mathbf{P}, \theta, \epsilon, i) = \frac{1}{N}\left| \mathbf{P_z} - \mathbf{Q_z}^{(i)}(\theta, \epsilon) \right|^2.
\end{equation}

\noindent An example experimental $P_z$ series matched to theoretical data is seen in Figure \ref{fig:particleB2match}.

\begin{figure}[H]
\centering
\includegraphics[width=0.7\textwidth]{figures/results/particleB/October_1_Particle_4_run_4match.pdf}
\caption{The upper figure shows the experimental $n_z$ peaks $\mathbf{P_z}$ versus the theoretical $Q_z$ for the best matching orbit. The lower plot shows where what section of the theoretical time series was used for matching, ie what $i$ from section \ref{sec:matchorbit} was chosen.}
\label{fig:particleB2match}
\end{figure}



\noindent As $\epsilon$ will not actually change for a single particle over $r$ different measurements $P_z^{(1)}, P_z^{(2)}, ..., P_z^{(r)}$, however the initial conditions $\theta$ and $i$ (the phase) will. So for each $\epsilon$ we find the best $\theta$ and $i$ using $\hat{S}(\mathbf{P}_j, \epsilon)$ 
\begin{equation}
\hat{S}(\mathbf{P}_j, \epsilon) =  \min(S(\mathbf{P}_j, \theta, \epsilon, i))
\end{equation}

\noindent and  then find the best $\epsilon$ using

\begin{eqnarray}
\epsilon_{best} = \min(\sum\limits_{j=1}^{r} \hat{S}(\mathbf{P}_j, \epsilon)^2).
\end{eqnarray}


It is important to note that this matching has a theoretical limitation. The asymmetry of our cylindrical particles is different from the asymmetry in the triaxial particles used in the theoretical models. The triaxial particles still have a rotational symetry for rotations of $\pi$ around any of the major/minor axes. This is not the case with the asymmetric cylindrical particles, which have no rotational symmetry. We therefore have to make the assumption that the difference between these two types of asymmetry is not significant and further theoretical work might reject this assumption.

% Refer to some matched plots once I have added those. I THINK basically everythig should already be in the saved plots and data folder. 