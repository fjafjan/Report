\clearpage
\mbox{}
\thispagestyle{empty}
\begin{abstract}
Orientational motion of axisymmetric particles in shear flow is important in a number of scientific fields. It was first studied theoretically by Jeffery, who showed that an axissymmetical particle in shear flow follows one of infinitely many periodic orbits depending on the initial condition of the particle. Later theoretical studies by Yarin \emph{et al.} suggested that asymmetric particles may follow periodic, quasi-periodic, or chaotic orbits depending only on the initial conditions. This thesis attempts to experimentally verify Yarin's predictions, using glass particles in a reversible Stokes flow in a microfluidic PDMS channel. An optical tweezer is used to control the initial conditions. An automatic tracking of the particles was developed, and a number of improvements were made compared to previous experiments by Einarsson \emph{et al.}~\cite{JonasExperiment}. We study the effects of asymmetry on the particles, and the transition from periodic to quasi-periodic orbits for different initial conditions, and for different degrees of asymmetry of the particles. Some measurements show good agreement with theoretical predictions, but there are also unexplained behaviours when the flow is reversed. The measurements were made in collaboration with Alexander Laas \cite{alexanderThesis}.
\end{abstract}

\newpage
\clearpage
\mbox{}
\thispagestyle{empty}
\section*{Acknowledgements}
I hereby wish to thank my supervisors Bernhard Mehlig and Dag Hanstorp for helping me through this journey. I want to thank girlfriend Callie Gibbons for supporting me through the work on this thesis and Alexander Laas for being a tireless and understanding co-worker. I want to thank all the contributors to the wealth of open source software which I have used to create everything from most of the software to more of the figures and of course this very report. 

\hfill Staffan Ankardal, Göteborg Sweden \today
