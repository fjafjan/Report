\begin{abstract}
The Jeffery orbits define the motion of axissymmetrical particles in shear flow and is thus important in the study of suspensions of particles. In this thesis I attempt to verify the equations of motion experimentally using glass particles in a reversible flow in a microfluidic PDMS channel and an optical tweezer. An automatic tracking of the particles was developed and a number of improvements were made compared to previous experiments such as by Einarsson et al\cite{JonasExperiment}. We study the effects of asymmetry on the particles and the transition from periodic to quasi-periodic orbits for different initial conditions of particles and for different degrees of asymmetry. One good match with theoretical results is found for some measurements, but there are some unexplained behaviours when the  flow is reversed.
\end{abstract}

\newpage
\clearpage
\mbox{}
\thispagestyle{empty}
\section*{Acknowledgements}
I hereby wish to thank my Supervisors Bernhard Mehlig and Dag Hanstorp for helping me through this journey. I want to thank girlfriend Callie Gibbons for supporting me through the work on this thesis and Alexander Laas for being a tireless and understanding co-worker. I want to thank all the contributors to the wealth of open source software which I have used to create everything from most of the software to more of the figures and of course this very report. 

\hfill Staffan Ankardal, Göteborg Sweden INSERT PROPER DATE 3/11/13

