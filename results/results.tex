The aim of these measurements is to show that the particles follow Jeffery orbits and to show that they exhibit quasi-periodic and periodic motion for different initial conditions. In order to show that there are no significant disturbances we examine the dynamics of the particles during reversal of the flow. If the particle reverts its dynamics perfectly there have been no noise or inertial effects affecting the rotational motion.

During the work of this thesis a large number of movies of particles have been recorded with gradual improvements to the setup primarily in terms of density matching, particle density (the number of particles per liquid volume) and  bubble elimination. In this section we present the data from two different particles. These were the only particles where there were reversals that retraced their movements closely for several stretches for both quasi-periodic and periodic orbits. One is referred to as particle A, the other as particle B. Particle A is approximately \unit[24]{$\mu m$} long so it has an aspect ratio $\lambda \approx 8$. Particle B is approximately \unit[20.5]{$\mu m$} long so it has an aspect ratio $\lambda \approx 7$. The measurements in this section were done together with Alexander Laas~\cite{alexanderThesis}.

We started each measurement at an approximate depth $D$ and at position $p_0 = (x_0, z_0)$ in the channel relative to the inlet on the right, closer to the pump. 
%We have assumed that the shear is entirely in the $y$ direction when doing 	our theoretical analysis. This means the flow profile has to be almost entirely flat in the $z$-direction which it only is if the particle is close to the centre. Variations do occur but when are less than \unit[10]{$\mu$m} as is seen in Figure \ref{fig:particleAsink}.

\newpage