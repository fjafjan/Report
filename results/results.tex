During the work of this thesis around 300 movies of particles have been recorded with gradual improvements to the setup in terms of density matching, particle density, bubble elimination etc. In this section we will present the data from three movies of two different particles. One referred to as particle A, the other as particle B. The measurements in this section were done together with Alexander Laas.

We started each measurement at an approximate depth $D$ and at position $p_0 = (x_0, z_0)$ in the channel relative to the inlet on the right, closer to the pump. Since we want the shear to be entire in the $y$ direction we would want to be as close to $z_0 = 0$ as possible, but variations less than \unit[1]{mm} should still have virtually identical shear.

In the time series plots below such as figure \ref{fig:particleA1}, the circular and star markers indicate the peaks used for estimating the winding number as explained in section \ref{sec:windingEstimation}. A circle is an estimated minima $m_i$, stars an estimated maxima $M_i$.

The aim of these measurements is to show that the particles will follow the Jeffery orbits, to show that they will exhibit quasi-periodic or periodic motion for different initial conditions. In order to show that there is no significant disturbance we also want to show that their motion after a reversal is reverted close to perfectly.
\newpage