


\begin{figure}[H]
\centering
\includegraphics[width=\textwidth]{figures/results/particleB/October1Particle4_runs3and5Orbits}
\caption{The phase map for $\lambda = 7$, the estimate of $\lambda$ from measurement was $6.7 \pm 0.1$. The highlighted orbits are from measurements 3 and 4.}
\label{fig:October1Particle4_runs3and5Orbits}
\end{figure}


\begin{figure}[H]
\centering
\includegraphics[width=\textwidth]{figures/results/particleB/October1Particle4runs2and2Orbits}
\caption{The S.O.S. for $\lambda = 7$ and $\epsilon=0.04$, the estimate of $\lambda$ from measurement was $6.7 \pm 0.1$. The highlighted orbits are from measurements 1 and 2.}
\label{fig:October1Particle4runs2and2Orbits}
\end{figure}


\subsection{Measurement 1}
\begin{figure}[H]
\begin{center}
\includegraphics[width=0.7\textwidth]{figures/results/particleB/October_1_Particle_4_run_2_winding.pdf}
\end{center}
\caption{The first two stretches match very well as well as the last two. In the reversal between these two there is a large change which begins at (1) where the flow is starting to revert. This reversal also occurs at the end of the channel closer to the pump. Starts at $ x_0 = 9.3 mm, z_0 = 35\mu m, D \approx 100\mu m$}
\label{fig:particleB1}
\end{figure}
	
\begin{figure}[H]
\begin{center}
\includegraphics[width=0.7\textwidth]{figures/results/particleB/October_1_Particle_4_run_2_D.pdf}
\end{center}
\caption{The speed of the particle against time and against position. In the plot against time there is an extra dip to 0 at around $t=150$ and $t=400$. This occurs only at the end of channel further away from the pump.}
\label{fig:particleB1speed}
\end{figure}

\subsection{Measurement 2}

\begin{figure}[H]
\begin{center}
\includegraphics[width=0.7\textwidth]{figures/results/particleB/October_1_Particle_4_run_4_A.pdf}
\end{center}
\caption{Mostly constant orbit for large $n_z$. The reversals at (1) and (3) both change the orbit slightly but the size of the change is exaggerated by $n_z$ being very close to 1. The actual change in orbit can be seen in figure \ref{fig:October1Particle4runs2and2Orbits} is not as large as this plot might indicate. There is missing data at (2) and (4) where the particle was lost in tracking for some time. Started at $x_0 = 28.6 mm, z_0 = 72\mu m, D = \approx 85\mu$ m. }
\label{fig:particleB2}
\end{figure}

\begin{figure}[H]
\centering
\includegraphics[width=0.7\textwidth]{figures/results/particleB/October_1_Particle_4_run_4_B.pdf}
\caption{In the upper figure we see the period }
\label{fig:particleB2sinking}
\end{figure}

\begin{figure}[H]
\centering
\includegraphics[width=0.7\textwidth]{figures/results/particleB/October_1_Particle_4_run_4match.pdf}
\caption{The upper figure shows the experimental $n_z$ peaks versus the theoretical ones for the best matching orbit. The lower plot shows where what section of the theoretical time series was used for matching, ie what $i$ from section \ref{sec:matchorbit} was chosen.}
\label{fig:particleB2match}
\end{figure}




\subsection{Measurement 3}
\begin{figure}[H]
\begin{center}
\includegraphics[width=0.7\textwidth]{figures/results/particleB/October_1_Particle_4_run_5_winding.pdf}
\end{center}
\caption{A circular quasi-periodic orbit. Started at $x_0 = 2.7 mm, z_0 = 76\mu m, D \approx 90\mu$m.}
\label{fig:particleB3}
\end{figure}




\subsection{Measurement 4}
\begin{figure}[H]
\begin{center}
\includegraphics[width=0.7\textwidth]{figures/results/particleB/October_1_Particle_4_run_3_winding.pdf}
\end{center}
\caption{While there is not large change in $n_z$ there seems to be some small variations that could correspond to a bent quasi-periodic orbit. Started at $x_0 = 12.9 mm, z_0 = 21\mu m, D \approx 85\mu$ m}
\label{fig:particleB4}
\end{figure}
