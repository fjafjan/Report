\subsection{Particle A}
Particle A is approximately \unit[24]{$\mu m$} long so it has a $\lambda$ close to 8 and the closest match for the asymetry is $\epsilon = 0.02$. \ref{fig:particleA1}


Using the algorithm described in section \ref{sec:matchorbit} we match the data from measurement 1 and 2 for particle A to find the closest matching $\epsilon$ and the best matching orbits. This is shown in Figure\ref{fig:particleAOrbitFit}. We can see that particle A is in a quasi-periodic circular orbit during measurement 1, it is matched to the lines indicating A B and C for the first, second and third stretches respectively.  After being shifted by  the optical tweezer particle A followed a periodic orbit during measurement 2. Measurement 2 is matched to the orbids D, E, F and G for the first, second, third and fourth stretches respectively. 

\begin{figure}[H]
\begin{center}
\includegraphics[width=\textwidth]{figures/results/particleA/October11Particle2_Orbits_Added.pdf}
\end{center}
\caption{Gray lines represent the phase map for $\lambda = 8$ and $\epsilon = 0.02$. The measured $\lambda$ was $8.2 \pm 0.1$. The orbits of the best fit theoretical fits to measurements are highlighted stretch by stretch. None of the orbits for this particle had any large variation despite being very close to $n_z=0$. The winding numbers are within 50\% of the estimates but both are too low, suggesting that the $\epsilon$ might be too low.}
\label{fig:particleAOrbitFit}
\end{figure}

\subsection{Particle B}

The same procedure is repeated for particle B using the data from measurement 1,2, 3 and 4. To make the graph less cluttered it is split into two figures, Figure \ref{fig:October1Particle4runs2and2Orbits} for measurement 1 and 2 and Figure \ref{fig:October1Particle4_runs3and5Orbits} for measurement 3 and 4. We can see that particle B is in a quasi-periodic circular orbit , it is matched to the lines indicating A B and C for the first, second and third stretches respectively.  After being shifted by  the optical tweezer particle A followed a periodic orbit during measurement 2. Measurement 2 is matched to the orbids D, E, F and G for the first, second, third and fourth stretches respectively. 

\begin{figure}[H]
\centering
\includegraphics[width=\textwidth]{figures/results/particleB/October1Particle4runs2and2Orbits}
\caption{Grey lines represent the poincare map for $\lambda = 7$ and $\epsilon=0.04$, the estimate of $\lambda$ from measurement was $6.7 \pm 0.1$. The highlighted orbits are the best fits to the stretches from measurements 1 and 2, A-D from measurement 1 and E-I from measurement 2.}
\label{fig:October1Particle4runs2and2Orbits}
\end{figure}


\begin{figure}[H]
\centering
\includegraphics[width=\textwidth]{figures/results/particleB/October1Particle4_runs3and5Orbits}
\caption{Gray lines represent the poincare map for $\lambda = 7$ and $\epsilon = 0.04$. the estimate of $\lambda$ from measurement was $6.7 \pm 0.1$ . The highlighted orbits are the best fits to the stretches from measurements 3 and 4.}
\label{fig:October1Particle4_runs3and5Orbits}
\end{figure}


